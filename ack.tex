
PVAS (Partitioned Virtual Address Space), created by Akio Shimada, is
the predecessor to PiP. He came up with the concept of employing
PIE. When Akio and I first met with Pavan Balaji, he already had a
keen understanding of the potential of PVAS. He also provided me with
some excellent inspiration for PiP. He suggested PiP as the name. Min
Si spent a lot of time assisting me with my PiP papers. Excellent
articles were written by Kaiming Ouyang to use PiP to enhance MPICH
performance. I was able to spend the majority of my time building PiP
thanks to Yutaka Ishikawa. Balazs Gerofi also provided me with some
insightful feedback on PiP. Noriyuki Soda created the Github actions
for testing PiP and really aided me in the development of PiP.  

Finally, I want to thank Fusa Hori, my wife, for letting me dedicate
my time to writing this book.
