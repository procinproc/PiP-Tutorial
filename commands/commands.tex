
\section{PiP Commands}

This section will describe on the PiP commands in the PiP
package. Some of them are already shown but explained very briefly. In
this section, details of PiP commands will be explained.

\subsection{\PIPKW{pip-man}}\label{sec:pip-man}

This command shows the PiP man pages. Although this is just a simple
shell script to run Linux's {\tt man} command with the {\tt MAN_PATH}
setting to the PiP man pages (if installed properly), users need not take 
care about the man path by using this command.

\subsection{\PIPKW{pipcc} and \PIPKW{pipfc}}

As already described in Section~\ref{sec:pipcc-exec}, \pipcomm{pipcc}
is the compiler script for compiling PiP programs for C and C++ and
\pipcomm{pipfc} is for Fortran.

The {\tt --which} option will show you the pass of the actual back-end
compiler. Or, users can specify the back-end compiler by setting the 
environment variable {\tt CC} for \pipcomm{pipcc} or {\tt FC} for
\pipcomm{pipfc}.

By default, \pipcomm{pipcc} and \pipcomm{pipfc} compile program to
produce the code which can run as a PiP root process and/or a PiP
task. Users may specify {\tt --piproot} option for PiP root only
program, or {\tt --piptask} option for PiP task only program. Indeed,
any PiP program compiled as PiP tasks can run as a PiP root too. Thus,
{\tt --piptask} option is equivalent to {\tt --pipboth} (to
be both root and task) option.

The actual compile options to be passed to the back-end compiler are
shown by 
specifying the {\tt --cflags} option and the link options are shown by
the {\tt --lflags} option. The {\tt --cflags} or {\tt --lflags}
disables the actual compiling and/or linking process.
All options and parameters not for \pipcomm{pipcc} and those Linux
commands cannot run as PiP 
programs. Additionally, any shell script (shebang) cannot run as a PiP
program. As shown in Listing~\ref{out:pip-check}, the {\tt ls} command
is implemented a shell script indeed.

\lstinputlisting[style=example, 
  caption={{\tt pip-check} - Execution Example}, label=out:pip-check]
                {commands/examples/pip-check.out}

The \pipcomm{pip-check} program does not guarantee a program to
run as a PiP program, even if it tells so. 

\subsection{\PIPKW{pip-exec}}

The \pipcomm{pip-exec} command is to invoke PiP tasks derived
from one program in the examples so far. However, \pipcomm{pip-exec}
can invoke multiple programs and all PiP tasks derived from those
programs share the same address space. To do this, programs are
separated by colon (:) (Listing~\ref{out:pip-exec}).

\lstinputlisting[style=example, 
  caption={{\tt pip-exec} - Execution Example}, label=out:pip-exec]
                {commands/examples/pip-exec.out}

\subsection{\PIPKW{pips}}\label{sec:pips}

\pipcomm{pips} is the command to output the list of currently running
PiP roots and PiP tasks in the similar way of what the Linux's {\tt
  ps} command does. Here is the example, running three (3)
\pipcomm{pip-exec} each of which execute {\tt a}, {\tt b}, or {\tt c}
PiP tasks. 

\begin{lstlisting}[frame=tRBl]
$ pips
PID   TID   TT       TIME     PIP COMMAND
18741 18741 pts/0    00:00:00 RT  pip-exec
18742 18742 pts/0    00:00:00 RG  pip-exec
18743 18743 pts/0    00:00:00 RL  pip-exec
18741 18744 pts/0    00:00:00 0T  a
18745 18745 pts/0    00:00:00 0G  b
18746 18746 pts/0    00:00:00 0L  c
18747 18747 pts/0    00:00:00 1L  c
18741 18748 pts/0    00:00:00 1T  a
18749 18749 pts/0    00:00:00 1G  b
18741 18750 pts/0    00:00:00 2T  a
18751 18751 pts/0    00:00:00 2G  b
18741 18752 pts/0    00:00:00 3T  a
\end{lstlisting}

As you see, this output looks very similar to the on of the {\tt ps}
command. The unfamiliar column titled {\tt PIP} represents
if this is a PiP root or PiP task (first character. '{\tt R}' means
root, the other numerical digit '{\tt 0-9}' means PiP task. The second
character represents PiP \pipterm{execution mode}, explained in
Section~\ref{sec:exec-mode}).

This \pipcomm{pips} command has many options. Refer PiP man page
(\ref{sec:pip-man}) for more details. 

\subsection{\PIPKW{pip-gdb}}\label{sec:pip-gdb}

\pipcomm{pip-gdb} is PiP-aware version of {\tt gdb} (GNU
debugger). PiP tasks are implemented as GDB's inferiors. Here is the
example of PiP-gdb debugging session.

\begin{lstlisting}[frame=tRBl]
(pip-gdb) info inferiors
  Num  Description              Executable
* 4    process 1904 (pip 2)     /somewhere/pip-task-2
  3    process 1903 (pip 1)     /somewhere/pip-task-1
  2    process 1902 (pip 0)     /somewhere/pip-task-0
  1    process 1897 (pip root)  /somewhere/pip-root
\end{lstlisting}

\subsection{\PIPKW{pip-mode} and \PIPKW{printpipmode}}\label{sec:pip-mode}

The \pipcomm{pip-mode} command is to set PiP execution mode and the
\pipcomm{printpipmode} outputs the current execution mode (refer to
Section~\ref{sec:exec-mode} and \ref{sec:proc-name}). 

\subsection{\PIPKW{libpip.so}}

The PiP library \pipterm{libpip.so} can also run as a program, showing the
information how the library was build and installed.

\lstinputlisting[style=example,basicstyle=\tiny\tt, 
  caption={{\tt libpip.so} - Execution Example}, label=out:libpip]
                {commands/examples/libpip.out}
