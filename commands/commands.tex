
\section{PiP Commands}

The PiP package's PiP commands are covered in this section. Some of
these have already been demonstrated but just briefly discussed. The
specifics of the PiP commands will be described in this section. 

\subsection{\PIPKW{pip-man}}\label{sec:pip-man}

The PiP man pages are displayed by this command. Although using this
command does not require users to be concerned with the {\tt MAN_PATH}, it
does execute the Linux {\tt man} command with the {\tt MAN_PATH}
option to the PiP man pages (if installed successfully).

\subsection{\PIPKW{pipcc} and \PIPKW{pipfc}}

\pipcmd{pipcc} is the compiler script for compiling PiP applications
for C and C++, and \pipcmd{pipfc} is for Fortran, as was already
mentioned in Section~\ref{sec:pipcc-exec}. 
The \pipcmdopt2{pipcc}{pipfc}{which} option will display the real
back-end computer's pass. For \pipcmd{pipcc} or \pipcmd{pipfc}, users
can also choose the back-end compiler by changing the environment
variable \pipenvC{pipcc}{CC} or \pipenvC{pipfc}{FC}.

By default, \pipcmd{pipcc} and \pipcmd{pipfc} compile programs to
create code that can execute as either a PiP task or the root
process. Users can specify the \pipcmdopt2{pipcc}{pipfc}{piproot} or
\pipcmdopt2{pipcc}{pipfc}{piptask} 
options to run an application that only 
supports PiP tasks or the PiP root. Any PiP software that has been
compiled as PiP tasks can, in fact, run as a PiP root. The
\pipcmdopt2{pipcc}{pipfc}{piptask} 
option is therefore the same as the \pipcmdopt2{pipcc}{pipfc}{pipboth}
option (to be both root and task).

The \pipcmdopt2{pipcc}{pipfc}{cflags} option displays the actual compilation
options that 
should be provided to the back-end compiler, and the
\pipcmdopt2{pipcc}{pipfc}{lflags} 
option displays the linker options. The actual compiling and/or
linking processes are disabled by the
\pipcmdopt2{pipcc}{pipfc}{cflags} or \pipcmdopt2{pipcc}{pipfc}{lflags} 
options.

\lstinputlisting[style=example, 
  caption={{\tt pip-check} - Execution Example}, label=out:pip-check]
                {commands/examples/pip-check.out}

The \pipcmd{pip-check} application can be used to determine whether a
file can be executed as a PiP program. A program may not actually run
as a PiP program even if it explicitly states that it will. Linux commands
cannot be run as PiP tasks, even if they are created as PIE programs. Any
shell script ({\it shebang}, the file with the extension
``{$\sharp$}!'') follows the same rules. Listing 1.30 shows an example
of a shell script that uses the ls command. 

\subsection{\PIPKW{pip-exec}}

The \pipcmd{pip-exec} command is used in the examples thus far to
execute PiP tasks that are derivations of a single program. Though all
PiP tasks created from those programs share the same address space,
\pipcmd{pip-exec} has the ability to invoke several programs. Programs
are divided in this way by colons (:) (Listing~\ref{out:pip-exec}).

\lstinputlisting[style=example, 
  caption={{\tt pip-exec} - Execution Example}, label=out:pip-exec]
                {commands/examples/pip-exec.out}

\subsection{\PIPKW{pips}}\label{sec:pips}

Similar to how the Linux ps command works, the \pipcmd{pips} command
outputs a list of all presently active PiP tasks and PiP roots. Here
is an example where three (3) \pipcmd{pip-exec} are launched, with
each one executing one of the PiP tasks (a, b, or c).

\begin{lstlisting}[frame=tRBl]
$ pips
PID   TID   TT       TIME     PIP COMMAND
18741 18741 pts/0    00:00:00 RT  pip-exec
18742 18742 pts/0    00:00:00 RG  pip-exec
18743 18743 pts/0    00:00:00 RL  pip-exec
18741 18744 pts/0    00:00:00 0T  a
18745 18745 pts/0    00:00:00 0G  b
18746 18746 pts/0    00:00:00 0L  c
18747 18747 pts/0    00:00:00 1L  c
18741 18748 pts/0    00:00:00 1T  a
18749 18749 pts/0    00:00:00 1G  b
18741 18750 pts/0    00:00:00 2T  a
18751 18751 pts/0    00:00:00 2G  b
18741 18752 pts/0    00:00:00 3T  a
\end{lstlisting}

As you can see, this output resembles the on of the {\tt ps} command
quite a bit. If this is a PiP root or PiP task, it is shown in the
unfamiliar {\tt PIP} column (first character). The other numerical
digit, ``0-9,'' stands for the PiP task, while ``R'' stands for
root. In Section~\ref{sec:exec-mode}, PiP execution 
mode is represented by the second character.

The \pipcmd{pips} command has many options. Refer PiP man page
(Section \ref{sec:pip-man}) for more details. 

\subsection{\PIPKW{pip-gdb}}\label{sec:pip-gdb}

PiP-aware {\tt gdb} (GNU debugger) is known as \pipcmd{pip-gdb}. PiP
tasks are introduced as inferiors of GDB. This is an illustration of a
PiP-gdb debugging session. Note that the \pipcmd{pip-gdb} can only
debug PiP tasks in the process mode.

\begin{lstlisting}[frame=tRBl]
(pip-gdb) info inferiors
  Num  Description              Executable
* 4    process 1904 (pip 2)     /somewhere/pip-task-2
  3    process 1903 (pip 1)     /somewhere/pip-task-1
  2    process 1902 (pip 0)     /somewhere/pip-task-0
  1    process 1897 (pip root)  /somewhere/pip-root
\end{lstlisting}

\subsection{\PIPKW{pip-mode} and \PIPKW{printpipmode}}\label{sec:pip-mode}

The \pipcmd{pip-mode} command is to set PiP execution mode and the
\pipcmd{printpipmode} outputs the current execution mode (refer to
Section~\ref{sec:exec-mode} and \ref{sec:proc-name}). 

\subsection{\PIPKW{libpip.so}}

The PiP library \pipterm{libpip.so} can also run as a program, showing the
information how the library was build and installed.

\lstinputlisting[style=example,basicstyle=\tiny\tt, 
  caption={{\tt libpip.so} - Execution Example}, label=out:libpip]
                {commands/examples/libpip.out}
