
\section{Execution Context}\label{sec:context}

Before I cover the following points, readers should be familiar with
the execution context under PiP. The general concept of the execution
context can be thought of as the CPU's state or the information in
hardware registers. On PiP, this definition might not be
adequate. Let's examine a case in point. Imagine that the same
program, which is operating as two PiP tasks, has a function called
{\tt foo()}. By providing the function pointer and leveraging the 
\pipfunc{pip_named_export} and \pipfunc{pip_named_import}, one PiP task
can call the function of another PiP task. An additional static
variable, let's call {\tt var}, is used by the function foo. When task
$A$ calls function {\tt foo()} of task $B$, the called function
accesses task B's variable rather than task A's
(Listing~\ref{prg:context} and \ref{out:context}). 

\lstinputlisting[style=program,
  caption={Calling a Function of Another Task},
  label=prg:context] {context/examples/context.c}

Sorry, this may detour off the main road, but in this example program,
\pipfunc{pip_named_export} and 
\pipfunc{pip_named_import} are called in a different way than they were
previously. Similar to Linux's \linuxfunc{printf}, the second and third
arguments to the \pipfunc{pip_named_export} and 
\pipfunc{pip_named_import} functions, respectively, are
really format strings followed by the argument(s) required by the
format.
 
\lstinputlisting[style=example, 
  caption={Function Call of Another Task - Execution}, label=out:context]
                {context/examples/context.out}

The execution context in a PiP environment may differ from that in a
common sense environment, as shown in Listing~\ref{out:context}, and
occasionally its behavior changes in a very subtle way. This occurs
pretty frequently in the PiP library and makes debugging quite
challenging. 

Furthermore, the CPU architecture and PIE implementation have a
significant impact on the relationship between static variables and
function addresses. For \AMD\ and \ARM, the aforementioned
description is accurate, but not on \INTEL. As a result, it is not
advised to do this.

\subsubsection*{Rationale}

Some readers may be curious as to why this occurs. Let me
elaborate. The address map contains this
ruse. Listing~\ref{out:glibc-segs} displays a section of the address
map with three jobs executing, with a focus on the Glibc  ({\tt
  /lib64/libc-2.28.so}) segments.  

\begin{lstlisting}[basicstyle=\tiny\tt, frame=tRBl, label=out:glibc-segs]
  ...
7ffff53f8000-7ffff55a4000 r-xp 00000000 fe:01 3049686    /lib64/libc-2.28.so
7ffff55a4000-7ffff57a4000 ---p 001ac000 fe:01 3049686    /lib64/libc-2.28.so
7ffff57a4000-7ffff57a8000 r--p 001ac000 fe:01 3049686    /lib64/libc-2.28.so
7ffff57a8000-7ffff57aa000 rw-p 001b0000 fe:01 3049686    /lib64/libc-2.28.so
  ...
7ffff69f6000-7ffff6ba2000 r-xp 00000000 fe:01 3049686    /lib64/libc-2.28.so
7ffff6ba2000-7ffff6da2000 ---p 001ac000 fe:01 3049686    /lib64/libc-2.28.so
7ffff6da2000-7ffff6da6000 r--p 001ac000 fe:01 3049686    /lib64/libc-2.28.so
7ffff6da6000-7ffff6da8000 rw-p 001b0000 fe:01 3049686    /lib64/libc-2.28.so
...
7ffff73d5000-7ffff7581000 r-xp 00000000 fe:01 3049686    /lib64/libc-2.28.so
7ffff7581000-7ffff7781000 ---p 001ac000 fe:01 3049686    /lib64/libc-2.28.so
7ffff7781000-7ffff7785000 r--p 001ac000 fe:01 3049686    /lib64/libc-2.28.so
7ffff7785000-7ffff7787000 rw-p 001b0000 fe:01 3049686    /lib64/libc-2.28.so
  ...
\end{lstlisting}

Glibc segments come in three sets. The final (readable and writable)
segment of each set is where the static variables are located. By
using the offset from the instruction (program counter relative
addressing mode) to the variable, an instruction can access a static
variable. In order to keep all offsets equal, the gap size between the
code segment (at the top of the set) and the variable segment (at the
bottom of the set) is crucial, and all gap sizes must be the
same. Variables and instructions are connected in \PIE\footnote{This is
not the case if not compiled as \PIE.} in this 
fashion, allowing \PIE\ applications and shared objects that were
created with the \PIC\ option to be loaded in any place.

Unfortunately, not all CPU architectures implement this addressing
mode. For example, \INTEL\ doesn't. This architecture sacrifices one
general purpose register in order to point the variable segment,
causing performance to suffer as a result. Moreover,
Listing~\ref{prg:context} program may display a distinct behavior.
