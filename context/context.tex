\section{Execution Context}

Before explaining the rest of the arguments, readers should know about
the execution context under PiP. The execution context can be defined
as the state of CPU, i.e., contents of hardware registers. On PiP,
this definition may not be enough. Let us have an example. Suppose
that the same program runs as two PiP tasks and this program has a
function {\tt foo()}. By passing the function pointer, by using the
\pipterm{pip_named_export()} and \pipterm{pip_named_import()}, one of
the PiP task can call the function of the other PiP
task. Additionally, this function accesses a static variable, say {\tt
  var}. If task $A$ calls function {\tt foo} of task $B$, then the
called function accesses the variable owned by task $B$, not $A$
(Listing~\ref{prg:context} and \ref{out:context}).

\lstinputlisting[style=program,
  caption={Function Call of Another},
  label=prg:context] {context/examples/context.c}

In this example program, sorry, this goes off the side road,
\pipterm{pip_named_export()} and 
\pipterm{pip_named_import()} are called differently than
before. The final argument of these functions is actually a format
string, just like {\tt printf()}, followed by argument(s) needed by
the format. 

\lstinputlisting[style=example, 
  caption={Function Call of Another - Execution}, label=out:context]
                {context/examples/context.out}

Thus, the execution context in PiP environment is different from the
one in common sense and some times can become very subtle. In the PiP
library, this happens quite often and makes debugging difficult.

Further, the situation described above, i.e., the association of
static variables and function address, deeply depends on the CPU
architecture and tool chain. The above description is true on  {\tt
  X86_64} and {\tt AArch64}, however, not true on {\tt X86_32}. Thus,
it is not recommended do this.

\subsubsection*{Rationale}

Some readers may wonder why this happens. Let me explain this. This
trick is hidden in the address map. Listing~\ref{out:glibc-segs} shows
a part of address map running three tasks, focusing on the Glibc ({\tt
  /lib64/libc-2.28.so}) segments. 

\begin{lstlisting}[basicstyle=\tiny\tt, frame=tRBl, label=out:glibc-segs]
  ...
7ffff53f8000-7ffff55a4000 r-xp 00000000 fe:01 3049686    /lib64/libc-2.28.so
7ffff55a4000-7ffff57a4000 ---p 001ac000 fe:01 3049686    /lib64/libc-2.28.so
7ffff57a4000-7ffff57a8000 r--p 001ac000 fe:01 3049686    /lib64/libc-2.28.so
7ffff57a8000-7ffff57aa000 rw-p 001b0000 fe:01 3049686    /lib64/libc-2.28.so
  ...
7ffff69f6000-7ffff6ba2000 r-xp 00000000 fe:01 3049686    /lib64/libc-2.28.so
7ffff6ba2000-7ffff6da2000 ---p 001ac000 fe:01 3049686    /lib64/libc-2.28.so
7ffff6da2000-7ffff6da6000 r--p 001ac000 fe:01 3049686    /lib64/libc-2.28.so
7ffff6da6000-7ffff6da8000 rw-p 001b0000 fe:01 3049686    /lib64/libc-2.28.so
...
7ffff73d5000-7ffff7581000 r-xp 00000000 fe:01 3049686    /lib64/libc-2.28.so
7ffff7581000-7ffff7781000 ---p 001ac000 fe:01 3049686    /lib64/libc-2.28.so
7ffff7781000-7ffff7785000 r--p 001ac000 fe:01 3049686    /lib64/libc-2.28.so
7ffff7785000-7ffff7787000 rw-p 001b0000 fe:01 3049686    /lib64/libc-2.28.so
  ...
\end{lstlisting}

There are three sets of Glibc segments. The static variables are
located on the last (readable and writable) segment of each set. A
static variable is accessed by an instruction using the offset from
the instruction (program counter relative addressing mode). Thus, the
gap size between the code segment (top of the set) and variable segment
(last of the set) is important to make all offsets constant and all gap
sizes must be the same. In this way, variables and execution code are
associated in PIE\footnote{This is not the case if not compiled as PIE.}.

Unfortunately, this addressing mode is not supported by all CPU
architectures\footnote{Listing~\ref{out:context} is obtained by
running the program on an {\tt X86_64} CPU}. For example, {\tt X86_32} 
does not. On this 
architecture, one general purpose register is sacrificed to point the
variable segment, resulting performance degradation by loosing one
general purpose register. And the program shown in
Listing~\ref{prg:context} exhibits differently. 
