
\section{Debugging Support}

While debugging PiP programs, some environment variable settings may
be helpful.  

\subsection{\PIPKW{PIP_STOP_ON_START}}

When this environment variable is set, the spawning PiP job is
suspended (by sending \linuxdef{SIGSTOP} shortly before the before
hook (Section~\ref{sec:hooks}) is called, or if before hook is not
specified, the starting function is called instead. The environment's
worth must conform to the following specification;

\begin{lstlisting}[frame=tb]
  PIP_STOP_START=[<script-file>]@<PIPID>
\end{lstlisting}

The {\option{PIPID} is the \PIPID\ of the suspending
task, and the optional {\option{script-file} is an executable
shell script to be run upon suspension. All PiP tasks that have been
spawned will be stopped if "PIPID" is -1. The "script" is called with
three parameters: the PID and PIPID of the suspended PiP task, then
the path to the task's program. Don't forget to set the executable bit
on this {\option{script-file}.

\lstinputlisting[style=program,
  caption={Stop-on-start Script Example},
  label=prg:stop-on-start] {debug/examples/onstart.cmd}

A sample of the PIP STOP ON START script is shown in
Listing~\ref{prg:stop-on-start}. In this case, the pips command is
used in place of a debugging command\footnote{
Although \linuxfunc{ptrace} (and other commands utilizing
\linuxfunc{ptrace}) were unable to run even with the {\tt
  —cap-add=SYS_PTRACE} Docker option, all examples in this paper were
tested in a Docker environment (I confirmed \linuxcmd{gdb}
worked). In this example, \pipcmd{pips} was used.}. It should be noted
that sending the \linuxdef{SIGSTOP} signal has already stopped the
target task. If you want to resume a task, you must somehow explicitly
provide the \linuxdef{SIGCONT} signal to the
task. Listing~\ref{out:stop-on-start} displays the outcome of running
this script file with the \pipenv{PIP_STOP_ON_START} environment
setting.  

\lstinputlisting[style=example, basicstyle=\tiny\tt, 
  caption={Stop-on-start Script Example - Execution},
  label=out:stop-on-start] {debug/examples/stop-on-start.out}

\subsection{\PIPKW{PIP_GDB_SIGNALS}}

When a signal provided in this environment variable is delivered to a
PiP task, PiP-gdb is called. The path to PiP-gdb must be specified by
the \pipenv{PIP_GDB_PATH} environment. Prior to calling PiP-gdb,
  actions related to \pipenv{PIP_SHOW_MAPS} and \pipenv{PIP_SHOW_PIPS}
  (both will be discussed below) will be taken. This environment's
  format is as follows: 

\begin{lstlisting}[frame=tb]
  PIP_GDB_SIGNALS=[ <SIGNAME> ] { "+"|"-" <SIGNAME> }
\end{lstlisting}

The following is a list of possible {\option{SIGNAME} values.
Here, "{\tt ALL}" refers to every signal listed in this table. The
plus ({\tt +}) and/or minus ({\tt -}) symbols can be used to
concatenate each signal name in this table. For instance, "{\tt
  ALL-SIGUSR1}" denotes all signals other than
\linuxdef{SIGUSR1}. \linuxdef{SIGUSR1}, \linuxdef{SIGUSR2}, and
linuxdef{SIGINT} are represented as "{\tt SIGUSR1+SIGUSR2+SIGINT}". 

\begin{table}[ht]
  \centering
  \caption{Possible Signal Names for \pipenv{PIP_GDB_SIGNALS}}
  \vspace{3mm}
  \tt
  \begin{tabular}{l}
    SIGHUP \\
    SIGINT \\
    SIGQUIT \\
    SIGILL \\
    SIGABRT \\
    SIGFPE \\
    SIGINT \\
    SIGSEGV \\
    SIGPIPE \\
    SIGUSR1 \\
    SIGUSR2 \\
    ALL \\
  \end{tabular}
\end{table}

\subsection{\PIPKW{PIP_SHOW_MAPS}}

The address map (Listing~\ref{out:maps}, for example) will be
displayed if \pipenv{PIP_SHOW_MAPS} environment is set to "{\tt on}"
and the signal specified by the \pipenv{PIP_GDB_SIGNALS} is provided.  

\subsection{\PIPKW{PIP_SHOW_PIPS}}

The \pipcmd{pips} command (Section~\ref{sec:pips}) is invoked to
display the status of the PiP tasks running in the same address space if
\pipenv{PIP_SHOW_PIPS} environment is set to "{\tt on}" and a signal 
specified by the \pipenv{PIP_GDB_SIGNALS} is delivered.

\subsection{\PIPKW{PIP_GDB_COMMAND}}

PiP-gdb will be invoked to work with this command file if the value of
the \pipenv{PIP_GDB_COMMAND} environment is set to a valid filename
containing a series of gdb commands. 
