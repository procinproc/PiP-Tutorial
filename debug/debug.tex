
\section{Debugging Support}

Some environment variable settings may help debugging PiP programs.

\subsection{\PIPKW{PIP_STOP_ON_START}}

This environment variable is to stop (by sending {\tt SIGSTOP} to
spawned PiP task just before calling the \pipterm{before hook} (if
any). The value of this environment must meet the following format;

\begin{lstlisting}[frame=tb]
  PIP_STOP_START=[<commandfile>]@<PIPID>
\end{lstlisting}

The optional {\tt\option{commandfile}} is a command file or shell
script to be executed on the suspension, and {\tt\option{N}} is the
PIPID to be suspended. If {\tt\option{N}} is -1, then all spawned PiP
tasks will be stopped. 

\subsection{\PIPKW{PIP_GDB_SIGNALS}}

This environment variable is to set the signals to trigger some
actions by specifying the \pipterm{PIP_SHOW_MAPS} and 
\pipterm{PIP_SHOW_PIPS}, followed by invoking PiP-gdb. The value of
this environment is as follows;

\begin{lstlisting}[frame=tb]
  PIP_GDB_SIGNALS=[ <SIGNAME> ] { "+"|"-" <SIGNAME> }
\end{lstlisting}

The possible {\tt\option{SIGNAME}} vale are listed below;

\begin{table}[ht]
  \centering
  \caption{Possible Signal Names for \pipterm{PIP_GDB_SIGNALS}}
  \vspace{3mm}
  \tt
  \begin{tabular}{l}
    SIGHUP \\
    SIGINT \\
    SIGQUIT \\
    SIGILL \\
    SIGABRT \\
    SIGFPE \\
    SIGINT \\
    SIGSEGV \\
    SIGPIPE \\
    SIGUSR1 \\
    SIGUSR2 \\
    ALL \\
  \end{tabular}
\end{table}

Here, {\tt ``ALL''} means all signals list in this table. Each signal
name in this table can be concatenated by using the plus (+) and/or
minus (-) symbols.  For example, ``{\tt ALL-SIGUSR1}'' indicates the
all signals excluding {\tt SIGUSR1}. ``{\tt SIGUSR1+SIGUSR2+SIGINT}''
represents {\tt SIGUSR1}, {\tt SIGUSR2} and {\tt SIGINT}. 

\subsection{\PIPKW{PIP_SHOW_MAPS}}

If \pipterm{PIP_SHOW_MAPS} environment is set to ``{\tt on}'' and the
a signal specified by the \pipterm{PIP_GDB_SIGNALS} is delivered, then
the address map (Listing~\ref{out:maps}, for example) will be shown. 

\subsection{\PIPKW{PIP_SHOW_PIPS}}

If \pipterm{PIP_SHOW_PIPS} environment is set to ``{\tt on}'' and the
a signal specified by the \pipterm{PIP_GDB_SIGNALS} is delivered, then
\pipterm{pips} command (Section~\ref{sec:pips}) is invoked to show the
status of the other PiP tasks in the same address space.

\subsection{\PIPKW{PIP_GDB_PATH} and \PIPKW{PIP_GDB_COMMAND} (\pipterm{
process mode} only)}

When \pipterm{PIP_GDB_PATH} is set to the path to \pipterm{pip-gdb}
a signal specified by the \pipterm{PIP_GDB_SIGNALS} is delivered, then
PiP gdb (Section~\ref{sec:pip-gdb}) will be invoked. If the value of
the \pipterm{PIP_GDB_COMMAND} environment is set to a valid filename
and if the filename contains some GDB commands, then PiP-gdb will be
invoked to work with this command file. 
