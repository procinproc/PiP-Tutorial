
\section{Execution Mode}\label{sec:exec-mode}

PiP library is made to function on Linux. According to
Section~\ref{sec:rationale}, it is very dependent on the
\linuxfunc{dlmopen} and \linuxfunc{clone} functions. Particularly, the
\linuxfunc{clone} function is called with a unique set of 
{\tt CLONE} flags. Many Linux variants exist, and some of them do not
support the specified {\tt CLONE} flag combination (for example,
McKernel\footnote{\url{https://github.com/ihkmckernel/mckernel}}). There
are two ways to execute PiP in such an environment: 
one involves executing \linuxfunc{clone} with a particular flag
combination, and the other involves calling \linuxfunc{pthread_create}
to start a PiP task while 
using the standard {\tt CLONE} flag combination. \pipterm{Process
  mode} refers to the former and \pipterm{pthread mode} to the
latter. Both modes keep the fundamental characteristics of the PiP,
including address space sharing and 
flexible privatization.

\subsection{Differences Between Two Modes}

{\tt CLONE} flag combinations ultimately determine the difference in the
PiP execution mode. The {\tt CLONE} flag values \linuxdef{CLONE_FS},
\linuxdef{CLONE_FILES}, \linuxdef{CLONE_SIGHAND}, and
\linuxdef{CLONE_THREAD} are reset, but \linuxdef{CLONE_VM} and
\linuxdef{CLONE_SYSVSEM} are set, in contrast to the 
pthread mode. 

\begin{table}[ht]
  \centering
  \caption{Differences between two modes}\label{tbl:mode-diff}
  \vspace{3mm}
  \begin{tabular}{c||c|c}
    \hline
    & Process Mode & Pthread Mode \\
    \hline
    \hline
    Address Space Sharing & yes & yes \\
    Variable Privatization & yes & yes \\
    File Descriptors (FDs) & not shared & shared \\
    \hline
  \end{tabular}
\end{table}

The primary distinctions between the two modes are shown in
Table~\ref{tbl:mode-diff}. There are a lot more variations, however
the PiP library offers mode-agnostic functions so that users can
develop PiP programs without worrying about the variations in mode.

\begin{table}[ht]
  \centering
  \caption{Mode-Agnostic Functions}\label{tbl:mode-agnostic}
  \vspace{3mm}
  \small
  \begin{tabular}{c||c|c||l}
    \hline
    Mode-Agnostic & Process Mode & Pthread Mode & \multicolumn{1}{c}{note} \\
    \hline
    \pipfunc{pip_exit()} & {\tt exit()} & {\tt pthread_exit()} &
    termination \\
    \pipfunc{pip_wait()} & {\tt wait()} & {\tt pthread_join()} &
    {\tiny wait termination} \\
    \pipfunc{pip_kill()} & {\tt kill()} & {\tt pthread_kill()} & send
    signal \\
    \pipfunc{pip_sigmask()} & {\tt sigprocmask()} & {\tt
      pthread_sigmask()} & signal mask \\
    \pipfunc{pip_signal_wait()} & sigwait() & sigwait() & wait signal
    \\
    \pipfunc{pip_yield()} & {\tt sched_yield()} & {\tt
      pthread_yield()} & yield \\
%%    \pipterm{pip_abort()} & {\tt abort()} & {\tt abort()} & kill all
%%    tasks \\
    \hline
  \end{tabular}
\end{table}

There are also predicate functions for users to know the current
mode listed in Table~\ref{tbl:mode-predicates}.

\begin{table}[ht]
  \centering
  \caption{Execution Mode Predicates}\label{tbl:mode-predicates}
  \vspace{3mm}
  \begin{tabular}{c|c}
    \hline
    Function name & \multicolumn{1}{c}{note} \\
    \hline
    \pipfunc{pip_is_threaded()} & if pthread mode \\
    \pipfunc{pip_is_shared_fd()} & if FDs are shared \\
    \hline
  \end{tabular}
\end{table}

In the current implementation, \pipfunc{pip_is_threaded} and
\pipfunc{pip_is_shared_fd} have the same meaning. There may be
situations in which those two terms have different meanings, so having
those functions is necessary. 

\subsection{How to Specify Execution Mode}

When calling \pipfunc{pip_init} or setting the \pipenv{PIP_MODE}
environment variable at runtime, the execution mode can be
selected. The function prototype 
for the \pipfunc{pip_init} shown below. Up until this point, the first
three arguments have been discussed.

\begin{lstlisting}[frame=tRBl]
int pip_init( int *pipidp,	[IN/OUT]
              int *ntasksp,     [IN/OUT]
              void **root_expp, [IN/OUT]
              int opts );       [IN]
\end{lstlisting}

The final opts parameter can take one of the following values:
\pipdef{PIP_MODE_PROCESS}, \pipdef{PIP_MODE_THREAD}, or both, or
zero. Zero has the same value as {\tt
  \pipdef{PIP_MODE_PROCESS}|\pipdef{PIP_MODE_PTHREAD}}. The
\pipenv{PIP_MODE} environment can be a string that contains the words
"{\tt process}" or "{\tt pthread}." The value of the \pipenv{PIP_MODE}
environment variable is checked when the opts value is zero or the
value of either oring the two. If no environment is specified, the PiP
library picks a suitable one. The environmental value and the ethical
worth cannot be at odds with one another. 

