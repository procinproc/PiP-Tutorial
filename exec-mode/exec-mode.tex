
\section{Execution Mode}\label{sec:exec-mode}

PiP library is designed to run on Linux. As described in
Section~\ref{sec:rationale}, it heavily depends on the {\tt
  dlmopen()} and {\tt clone()}. Especially, the {\tt clone()} is
called with a rare combination of {\tt CLONE} flags. There are many
Linux variants and some of them do not support such a {\tt CLONE} 
flag combination (for example, McKernel). To run PiP on
such environment, there are two PiP 
execution modes, one for calling {\tt clone()} with the special flag
combination and another for calling {\tt pthread_create()} (using the
normal flag combination) to spawn a PiP task. The former is called
\pipterm{process mode} and latter is called \pipterm{pthread mode}. In
either mode, the PiP's basic nature, sharing address space and
variable privatization are preserved. 

\subsection{Differences Between Two Modes}

The difference of the PiP \pipterm{execution mode} ends up with the
difference of the {\tt clone()} flag combination. Unlike the
\pipterm{pthread mode}, the {\tt CLONE} flags of {\tt CLONE_FS}, {\tt
  CLONE_FILES}, {\tt CLONE_SIGHAND} and {\tt CLONE_THREAD} are reset,
{\tt CLONE_VM} and {\tt CLONE_SYSVSEM} is set. 

\begin{table}[ht]
  \centering
  \caption{Differences between two modes}\label{tbl:mode-diff}
  \vspace{3mm}
  \begin{tabular}{c||c|c}
    \hline
    & Process Mode & Pthread Mode \\
    \hline
    \hline
    Address Space Sharing & yes & yes \\
    Variable Privatization & yes & yes \\
    File Descriptors (FDs) & not shared & shared \\
    \hline
  \end{tabular}
\end{table}

Table~\ref{tbl:mode-diff} shows the major differences between the two
modes. There are many other differences, though, PiP library provides
mode-agnostic functions so that users can write PiP programs without
care of the mode differences. 

\begin{table}[ht]
  \centering
  \caption{Mode-Agnostic Functions}\label{tbl:mode-agnostic}
  \vspace{3mm}
  \small
  \begin{tabular}{c||c|c||l}
    \hline
    Mode-Agnostic & Process Mode & Pthread Mode & \multicolumn{1}{c}{note} \\
    \hline
    \pipterm{pip_exit()} & {\tt exit()} & {\tt pthread_exit()} &
    termination \\
    \pipterm{pip_wait()} & {\tt wait()} & {\tt pthread_join()} &
    {\tiny wait termination} \\
    \pipterm{pip_kill()} & {\tt kill()} & {\tt pthread_kill()} & send
    signal \\
    \pipterm{pip_sigmask()} & {\tt sigprocmask()} & {\tt
      pthread_sigmask()} & signal mask \\
    \pipterm{pip_signal_wait()} & sigwait() & sigwait() & wait signal
    \\
    \pipterm{pip_yield()} & {\tt sched_yield()} & {\tt
      pthread_yield()} & yeild \\
    \pipterm{pip_abort()} & {\tt abort()} & {\tt abort()} & kill all
    tasks \\
    \hline
  \end{tabular}
\end{table}

There are also predicate functions for users to know the current
mode listed below;

\begin{table}[ht]
  \centering
  \caption{Execution Mode Predicates}\label{tbl:mode-predicates}
  \vspace{3mm}
  \begin{tabular}{c|c}
    \hline
    Function name & \multicolumn{1}{c}{note} \\
    \hline
    \pipterm{pip_is_threaded()} & if pthread mode \\
    \pipterm{pip_is_shared_fd()} & if FDs are shared \\
    \hline
  \end{tabular}
\end{table}

\subsection{Specifying Execution Mode}

The execution mode can be specified when to call \pipterm{pip_init()}
and/or setting the \pipterm{PIP_MODE} environment variable at
run-time. Below is the function prototype of the
\pipterm{pip_init()}. The first three arguments are already described
up until now.

\begin{lstlisting}[frame=tRBl]
int pip_init( int *pipidp,	[IN/OUT]
              int *ntasksp,     [IN/OUT]
              void **root_expp, [IN/OUT]
              int opts );       [IN]
\end{lstlisting}

The possible values of the last {\tt opts} argument are one of 
\pipterm{PIP_MODE_PROCESS} and \pipterm{PIP_MODE_PTHREAD}, or ORing
the both. The value of zero is equal to {\tt
  PIP_MODE_PROCESS|PIP_MODE_PTHREAD}. As for the \pipterm{PIP_MODE}
environment, it can be a string of ``\pipterm{PIP_MODE_PROCESS}'' or
``\pipterm{PIP_MODE_PTHREAD}.''
When the {\tt opts} value is zero or {\tt OR}ing the both, then the
\pipterm{PIP_MODE} envbironment value is checked. If the environment
is not set, then the PiP library chooses the possible one.

