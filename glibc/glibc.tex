
\section{Issues related to Linux Kernel, Glibc and Tools}

This section will explain about the issues when implementing PiP. 

\subsection{Loading a Program}

Before going into the details about the Glibc issues when implementing
PiP, readers should understand how a program is loaded into
memory. This subsection describes only about the program loading
procedure of Linux, apart from PiP implementation.

When the Linux's \linuxfunc{execve()} system call to run a program, the
Linux kernel open and read the executable file, searching the ELF
section named ``{\tt .interp}.''

\lstinputlisting[style=example, basicstyle=\tiny\tt, 
  caption={{\tt .interp} Section of the {\tt ps} command},
  label=out:interp] {glibc/examples/interp.out}

Listing~\ref{out:interp} shows the value of the {\tt .interp} section,
{\tt /lib64/ld-linux-x86-64.so.2}. The Linux kernel
invokes the loader specified by the {\tt .interp} section and asks the
loader to load a program specified by the \linuxfunc{execve()} parameter.
Then the loader load the program and additionally load and link the shared
libraries required to run the program. Once everything is loaded, the
loader jumps into the starting function defined in Glibc to initialize
Glibc and finally user-defined {\tt main()} function is called.

The program loader, often simply called {\tt ld-linux.so}, is loaded
once per address space and kept in memory until the end of the
process (see also Listing~\ref{out:maps}). This is responsible for any
loading process by resolving the 
external symbol references. The Glibc functions defined in the {\tt
  libdl.so} ({\tt -ldl}), such as \linuxfunc{dlopen()}, {\tt
  dlmopen()}, \linuxfunc{dlsym()} and so on, are just API 
and their functional bodies exist in this program loader.

\subsection{Glibc}

PiP provides a new execution model which cannot be categorized into
neither the process model nor the thread model. In this new model,
although its name is not yet given, tasks share the same address
space like the thread model, but maintaining the variable
privatization like the process model. This execution
model is novel and not yet recognized by most of the tool chains
provided by Linux and others. Indeed, the most of the time to develop
PiP was devoted to find niches in Glibc.

\subsubsection{PiP Task is Unable to Spawn PiP Task}

As described in Section~\ref{sec:rationale}, the order of calling
\linuxfunc{dlmopen()} and {\tt clone()} is important. This restriction
also means that a PiP task cannot spawn a PiP task as a child of
spawning PiP task because this breaks the restriction. Thus, the
current PiP implementation inhibits for a PiP task to call
\pipfunc{pip_spawn()}. 

\subsubsection{Recycling PiP Tasks}\label{sec:recycle}

As far as I tested, the resources; name space, loaded PIE program, and shared
libraries required by the PIE program, are not released by calling the
\linuxfunc{dlclose()}. I believe this issue can be fixed by patching the
Gilbc, however, I decided no to do so. Thus, once a PiP task is
created, then the \PIPID\ of the task will not be recycled even the PiP
task terminates. The reason of my decision will also be
discussed in Section~\ref{sec:retrive}.

\subsubsection{Number of name spaces}\label{sec:name-space}

The number of names spaces which the \linuxfunc{dlmopen()} can create is
hard-coded as 16. Considering PiP tasks run in parallel and the number
of CPU cores nowadays, this number of 16 is apparently too small. The
PiP package provides PiP-glibc where the number of name spaces is
increased, up to 300 PiP tasks\footnote{Once I asked Glibc
development members to increae the size, but they did not accept my
opinion. Refer
\url{https://sourceware.org/bugzilla/show_bug.cgi?id=23978}}.  

The name space table resides in the {\tt ld-linux.so} and this means
that the {\tt .interp} ELF section of PiP programs must be changed so
that the program is loaded by the new {\tt ld-linux.so}. This can be
done by specifying {\tt --dynamic-linker} option of the GNU linker and
the \pipcmd{pipcc} and \pipcmd{pipfc} do this.

The name space table resides at the top of a structure in {\tt
  ld-linux.so}. Some Glibc functions refer to the members in this
structure directly. This causes another problem. Once the size of the
name space table is changed, the addresses of the other members in the
same structure are also changes. As described, only one {\tt
  ld-linux.so} can be loaded in an address space. As a result, all PiP 
programs sharing the same address must be linked with the same Glibc. 

\subsubsection{PiP-gdb}

The {\tt ld-linux.so} embeds some information for debugging into the
loaded program. Unfortunately, I found that this code fragment
resides on the pass calling the {\tt ld-linux.so} from the
top (by the kernel), not on the pass called from \linuxfunc{dlopen()}
  and {\tt dlmopen()}\footnote{I guess the loaded code by using
  \linuxfunc{dlopen()} or \linuxfunc{dlmopen()} cannot be debugged.}. The
  patched 
  PiP-glibc fixed this issue. Thus, the \pipcmd{pip-gdb} command
  (Section~\ref{sec:pip-gdb}) can only work with 
  the PiP programs linked with the patched PiP-glibc.

\subsubsection{Global lock}

Most programs are linked with Glibc and PiP programs are no
exception. PiP allows to run multiple PiP programs in the same address
space. This means that each PiP task has its own Glibc. And the
simultaneous calls of some Glibc functions may not work because of a
race condition. 

To avoid this condition, PiP library provides the functions,
\pipfunc{pip_glibc_lock()} and \pipfunc{pip_glibc_unlock()}, to 
serialize the Gilbc function calls. The following Glibc functions are
wrapped by PiP library to introduce the lock and users do not have to
care the race. 

\begin{table}[ht]
  \centering
  \caption{Glibc functions wrapped by PiP library}\label{tbl:pip-wrapper}
  \vspace{3mm}
  \tt
  \begin{tabular}{lll}
    \hline
    dlsym	&
    dlopen	&
    dlmopen	\\
    dlinfo	&
    dlclose	&
    dlerror	\\
    dladdr	&
    dlvsym	&
    getaddrinfo	\\
    freeaddrinfo &
    gai_strerror &
    pthread_create \\
    \hline
    pthread_exit \\
    \hline
    malloc 	&
    free	&
    calloc	\\
    realloc	&
    memalign 	&
    posix_memalign \\
    \hline
  \end{tabular}
\end{table}

The functions \linuxfunc{pthread\_exit()} and below in this table have
another reason to have function wrappers. The wrapping reason of
\linuxfunc{pthread\_exit()} will be explained in the 
Section~\ref{sec:exec-mode} and the reason of wrapping {\tt malloc}
routines will be explained in Section~\ref{sec:malloc}. 

These listed functions may not be complete. There can
be a case where some other Glibc functions may suffer from the
race condition. This problem can be avoided by introducing the above
locking functions. This lock can be used recursively and users can
avoid deadlock situation easily. 

\subsubsection{Constructors and Destructors}

The constructors and destructors are used in C++
programs. Constructors and destructors are list of
functions. Generally, constructor functions are called just before the 
program begins, and destructors functions are called when the program
is about to exit.

In PiP, the behavior of the constructors and destructors is
somewhat different. To explain this, I should start explaining how the
constructors and destructors are implemented in general. The
constructor functions are listed in the {\tt .init_array} section of an ELF
file. The destructor functions are listed in the {\tt .fini_array}
section. Constructors are called when {\tt ld-linux.so} finishes
loading and linking objects. Destructors are called when
\linuxfunc{dlclose()} is called. 

Now back to PiP. Again, constructors are called inside
of the call of \linuxfunc{dlmopen()} when spawning a PiP task. The
\linuxfunc{dlmopen()} is called by the PiP root process. Thus, the
constructors of a program are called by the root. Here is the example;

\lstinputlisting[style=program,
  caption={Constructors and Destructors},
  label=prg:const-dest] {glibc/examples/hello.cpp}

Listing~\ref{prg:const-dest} is a C++ program having a constructor and
destructor. When the constructor of this program is called, the PiP
library is not yet initialized, and the \pipfunc{pip_get_pipid()}
return an error (\linuxdef{EPERM}). So, the function {\tt pipidstr()} takes
care of this situation. Listing~\ref{out:const-dest} shows the
executoin example of this program. As shown, the PIDs output by the
constructors are not the same with the ones of the PiP tasks. 

\lstinputlisting[style=example, 
  caption={Constructors and Destructors - Execution},
  label=out:const-dest] {glibc/examples/hello.out}

\subsubsection{{\tt LD_PRELOAD}}

\linuxenv{LD_PRELOAD} only works with PiP root, not PiP tasks. This is
because \linuxfunc{dlmopen()} simply ignores the \linuxenv{LD_PRELOAD}
environment setting.

\subsubsection{Shared Objects}

Some shared objects, such as GCC related runtime libraries, must be
located in the same directory where the {\tt ld-linux.so}
does. The \pipcmd{piplnlibs} shell script found in the PiP-glibc
package makes symbolic links of the shared objects in the {\tt /lib64}
directories to meet with the restriction.

\subsubsection{Loading Program by {\tt dlmopen()}}

The Glibc in CentOS/RedHat 8 (and possibly newer ones) does not allow
to load a program by the \linuxfunc{dlmopen()} function\footnote{Refer
\url{https://sourceware.org/bugzilla/show_bug.cgi?id=11754\#c15}. I
tested this situation but I cannot find this problem with PiP.}. The
\pipcmd{pip-unpie} program is to cheat this Glibc restriction. This
program is automatically executed by the \pipcmd{pipcc} or
\pipcmd{pipfc} when creating a PiP executable, and not to be invoked
by users directly.

\subsection{Glibc {\tt RPATH} Setting}

When using Spack\footnote{\url{PIPKW{pip-unpie} program}}, it
automatically adds {\tt RPATH}s for every program and PiP is no
exception. A problem arises when to install the PiP-glibc by using
Spack. When the PiP-glibc is built by using Spack, Spack adds the {\tt
  RPATH} setting to the compiled Glibc, but it is not allowed in
CentOS/Redhat 8 to load Glibc with the {\tt RPATH} setting. To avoid
this, PiP-gilbc has a program (\pipcmd{annul_rpath}) to unset
the {\tt RPATH} setting of the compiled Glibc.

\subsection{Linux}

\subsubsection{Heap Segment}\label{sec:heap}

There is another issue which comes from Linux kernel, not from
Glibc. Before explaining this issue, let us start from the how an
address space is composed. 

\lstinputlisting[style=example, basicstyle=\tiny\tt, 
  caption={A Memory Map Example},
  label=out:maps] {glibc/examples/maps.out}

Listing~\ref{out:maps} shows an example of the output of doing ``{\tt
  cat /proc/\option{PID}/maps}.'' Here, ``{\tt pip-exec -n 2
  ./a.out}'' was executed, resulting one \pipcmd{pip-exec} process
and two {\tt ./a.out} tasks. The file, {\tt
  /proc/\option{PID}/maps}, lists all memory
segments in an address space of the process {\tt \option{PID}}
usually. A loaded shared object has consecutive three or four segments; 
executable, gap (not accessible, if any), constants and data. 
The rightmost column of a line indicates the {\tt
  mmap()}ed filename, the second from the left column indicates the
permission of the memory segment. '{\tt r}' is readable, '{\tt w}' is
writable, '{\tt x}' is executable and '{\tt p}' is private
(copy-on-write). There are also some special segments whose filename
is in a pair of square brackets; {\tt [stack]}, {\tt [heap]}, and so
on. These are created by the Linux kernel for special purposes as
their names suggest. The segments having no filename are created by
the \linuxfunc{mmap()} system call. 

Remember, this is the address space of running one PiP root and two PiP
tasks, resulting to have all the segments of the three tasks.
Note that the all three tasks have
exactly the same {\tt /proc/\option{PID}/maps} content, and
there are three sets of a shared library and only one set of {\tt
  ld-linux.so} ({\tt ld-2.28.so}) can be seen in
Listing~\ref{out:maps}.

Usually, the heap segment, mainly used by \linuxfunc{malloc()}, exists only
one per address space. As shown in this example, there is only one heap
segment, meaning the heap segment is shared by three tasks (one for
root and two for PiP tasks). 

The size of the heap segment can be increased or decreased by calling
the \linuxfunc{brk()} system call. Most cases, there are two calls of
\linuxfunc{brk()} to allocate or deallocate heap memory, one for
obtaining the 
current heap end address and another for setting the new heap end 
address. This is exactly what the Glibc's \linuxfunc{sbrk()}
does. Apparently, this API is not thread-safe at all, and thus, 
the shared heap memory cannot be used safely by PiP tasks.

Fortunately, the \linuxfunc{malloc} routines in Glibc is designed to check if
there are two or more name spaces and if so they do not use the
\linuxfunc{brk()} system call, use {\tt mmap()} instead. So, the Glibc
\linuxfunc{malloc} routines can work with PiP without any
problem. However, if some other routines use the \linuxfunc{brk()}
system call (or \linuxfunc{sbrk()} Glibc function), for example,
replacing the Glibc \linuxfunc{malloc} routines with some other
\linuxfunc{malloc} implementation, then this shared heap may result in a 
problem.

\subsubsection{Core File}

Suppose that we have a catastrophic situation and all PiP tasks and
their root process dump core files of their own. On the current Linux,
a core file is associated with a process (including threads inside of
it). Thus, each PiP task and the root may produce core, resulting to
have many core files. Here, the address space of them are shared and
the created core files and all of them are almost the same excepting
the CPU state. 

Let me explain this with an example. Suppose that we have PiP task $A$
and $B$ running on the same address space of the root $P$, and an
error happens resulting all $P$, $A$, and $B$ produce core
files. There can be a small time difference when to produce each core
file. When the first core file, of $A$ for instance, is being created,
the other $P$ and $B$ are still running and the memory of the shared 
address space can be altered by those running tasks.
\linuxcomm{gdb}, however, assumes that a core file is a consistent snapshot
of memory and CPU state. The above PiP situation breaks this
assumption. If $B$ produces another core file, may or may not be
caused by the error on $A$, the same situation can happen.
Thus, the \pipcmd{pip-gdb} command (Section~\ref{sec:pip-gdb}) does
not support for debugging from a core file. To solve this issue,
PiP-aware OS kernel to have the consistent core files is needed. 

\subsection{Tools}

As described, PiP sets a special combination of the \linuxfunc{clone()}
flags. As a result of this, some tools do not work. Here is the list
of tools which are known to work or not at the time of this
writing\footnote{\linuxcomm{ltrace} depends on its version}. 

\begin{table}[ht]
  \centering
  \caption{Compatibility of Tools}
  \vspace{3mm}
  \begin{tabular}{c|c}
    \hline
    Compatible & Incompatible \\
    \hline
        {\tt strace} & {\tt valgrind} \\
        {\tt ltrace} \\
        \hline
  \end{tabular}
\end{table}
