
\section{Linux Kernel, Glibc, and Tools Problems}

The problems of adopting PiP will be discussed in this section. 

\subsection{Loading a Program}

Readers should have a basic understanding of how programs are loaded
into memory before reading more in-depth about the Glibc difficulties
when implementing PiP. Except the PiP technology, this paragraph only
discusses the Linux application loading process.

When a program is launched on Linux using the \linuxfunc{execve}
system call, the Linux kernel opens and reads the executable file
while looking for the "{\tt .interp}" ELF section.

\lstinputlisting[style=example, basicstyle=\tiny\tt, 
  caption={{\tt .interp} Section of the {\tt ps} command},
  label=out:interp] {glibc/examples/interp.out}

The value of the {\tt .interp} section, /lib64/ld-linux-x86-64.so, is
displayed in Listing~\ref{out:interp}. To load a program supplied by
the \linuxfunc{execve} argument, the Linux kernel invokes the loader
specified by the {\tt .interp} section.
The program is then loaded by the loader, along with the shared
libraries needed to run it. The loader then calls the user-specified
\main/ function after Glibc has been initialized by the starting
function defined in Glibc.

Once per address space, the program loader, which is frequently
referred to as {\tt ld-linux.so}, is loaded and held in memory until
the process finishes (see also Listing~\ref{out:maps}). By resolving
external symbol references, it is in charge of any loading
procedure. The Glibc functions, such as \linuxfunc{dlopen},
\linuxfunc{dlmopen}, \linuxfunc{dlsym}, and others, defined in the
{\tt libdl.so} ({\tt -ldl}), are only API, and this program
loader contains their functional bodies.

\subsection{Glibc}

PiP offers a novel execution model that defies classification as
either a process model or a thread model. While the name of this new
model has not yet been revealed, tasks in it maintain variable
privatization similar to the process model while sharing the same
address space as threads. The majority of the tool chains offered by
Linux and others do not yet understand this execution style because it
is new. In fact, the majority of the work spent developing PiP was
spent trying to locate Glibc niches. 

\subsubsection{PiP Task is Unable to Spawn PiP Task}

The sequence in which \linuxfunc{dlmopen} and \linuxfunc{clone} are
called matters, as discussed in Section~\ref{sec:rationale}. Due to
this restriction, a PiP task cannot launch another PiP task as a child
task because doing so violates the constraint. As a result, calling
\pipfunc{pip_task_spawn} or \pipfunc{pip_spawn} from a PiP task is
prevented by the present PiP implementation. 

\subsubsection{Recycling PiP Tasks}\label{sec:recycle}

According to my testing, calling the \linuxfunc{dlclose} function does
not release the name space, loaded PIE application, or shared
libraries used by the PIE program. I think fixing the Gilbc might
solve this problem, but I've chosen against it. Hence, even if a PiP
task ends after being established, the task's \PIPID\ will not be
recycled. The rationale for my choice will also be covered in
Section~\ref{sec:retrive}. 

\subsubsection{Number of name spaces}\label{sec:name-space}

The hard-coded limit for the number of names spaces that
\linuxfunc{dlmopen} can create is 16. This number of 16 appears to be
inadequate when taking into account the parallel execution of PiP jobs
and the current CPU core count. The PiP package offers \pipglibc,
which offers up to 300 PiP tasks and more name spaces\footnote{Once I
  asked Glibc development members to increae the size, but they did
  not accept my opinion. Refer
  \url{https://sourceware.org/bugzilla/show_bug.cgi?id=23978}}.

As the name space table is located in the {\tt ld-linux.so}, PiP
programs' {\tt .interp} ELF section needs to be modified in order for
the program to be loaded by the patched {\tt ld-linux.so}. The GNU
linker's {\tt \dash2{}dynamic-linker} option can be used to accomplish this,
and the \pipcmd{pipcc} and \pipcmd{pipfc} do this.

In {\tt ld-linux.so}, the name space table is found at the top of a
structure. A few Glibc internal functions make direct references to the
structure's members. This results in yet another issue. The addresses
of the other members of this structure are likewise updated once
the name space table's size is altered. One {\tt ld-linux.so} can only
be loaded at a time in an address space, as stated. As a result, the same
Glibc must be associated with each PiP program sharing the same
address.

\subsubsection{\pipgdb}

The loaded application has information for debugging that is embedded
by the {\tt ld-linux.so} file. Unfortunately, I discovered that this
code fragment is located on the pass that the kernel uses to call
ld-linux.so, not on the pass that was called from \linuxfunc{dlopen}
and \linuxfunc{dlmopen}. This problem was fixed with the \pipglibc\,
the patched Glibc. As a result, only PiP applications linked with
\pipglibc\ can be used with the \pipcmd{pip-gdb} command
(Section~\ref{sec:pip-gdb}).  

\subsubsection{Global lock}

The majority of programs, including PiP programs, are linked using
Glibc. Several PiP programs can execute simultaneously in the same
address space thanks to PiP. As a result, every PiP task has a unique
Glibc. Moreover, a race condition may prevent several Glibc functions
from working when called simultaneously.  

PiP library offers the routines \pipfunc{pip_glibc_lock} and
\pipfunc{pip_glibc_unlock} to serialize the Gilbc function calls in
order to prevent this situation. The PiP library wraps the following
Glibc routines so that the lock can be introduced and users don't have
to worry about the race.

\begin{table}[ht]
  \centering
  \caption{Glibc functions wrapped by PiP library}\label{tbl:pip-wrapper}
  \vspace{3mm}
  \tt
  \begin{tabular}{lll}
    \hline
    dlsym	&
    dlopen	&
    dlmopen	\\
    dlinfo	&
    dlclose	&
    dlerror	\\
    dladdr	&
    dlvsym	&
    getaddrinfo	\\
    freeaddrinfo &
    gai_strerror &
    pthread_create \\
    \hline
    pthread_exit \\
    \hline
    malloc 	&
    free	&
    calloc	\\
    realloc	&
    memalign 	&
    posix_memalign \\
    \hline
  \end{tabular}
\end{table}

This table's functions \linuxfunc{pthread_exit} and below have
function wrappers for still another reason. The malloc functions will
be wrapped for the reasons that will be explained in
Section~\ref{sec:malloc}, and \linuxfunc{pthread_exit} will 
be wrapped for the reasons that will be explained in
Section~\ref{sec:exec-mode}. 

These stated functions might not be all of them. Other Glibc functions
may experience the race condition in some circumstances. The
aforementioned locking functions can be added to eliminate this
issue. Users can easily prevent deadlock situations by using this lock
in a recursive fashion.

\subsubsection{Constructors and Destructors}

Programs written in C++ employ constructors and
destructors. Destructor functions are typically called as the program
is about to end, whereas constructor functions are typically called
right before the program starts.  

The behavior of the constructors and destroyers differs slightly in
PiP. I should begin by describing the generic implementation of
constructors and destructors in order to clarify this. The {\tt
  .init_array} section of an ELF file contains a list of the
constructor functions. The {\tt .fini_array} section contains a list
of the destructor functions. When {\tt ld-linux.so} has finished
loading and linking objects, constructors are called. When
\linuxfunc{_exit} or \linuxfunc{dlclose} is called, destructors are
invoked. 

Back to PiP now. When starting a PiP process, constructors are once
more called inside of the \linuxfunc{dlmopen} call. The PiP root
process calls the \linuxfunc{dlmopen} function. Hence, the root of a
program calls the constructors. Here is the example;

\lstinputlisting[style=program,
  caption={Constructors and Destructors},
  label=prg:const-dest] {glibc/examples/hello.cpp}

A C++ program with a constructor and destructor is shown in
Listing~\ref{prg:const-dest}. The PiP library has not yet been
initialized when the constructor of this program is invoked, and
\pipfunc{pip_get_pipid} returns an error ({\tt EPERM}). Hence, the
{\tt pipidstr()} function handles this circumstance. The 
executable example of this program is shown in
Listing~\ref{out:const-dest}. As can be seen, the PIDs produced by the
constructors differ from those of the PiP tasks.

\lstinputlisting[style=example, 
  caption={Constructors and Destructors - Execution},
  label=out:const-dest] {glibc/examples/hello.out}

\subsubsection{{\tt LD_PRELOAD}}

PiP tasks cannot be used with \linuxenv{LD_PRELOAD}; only PiP root
may. This is due to the fact that dlmopen() flatly disregards the
\linuxenv{LD_PRELOAD} environment variable.

\subsubsection{Shared Objects}

Certain shared objects, including runtime libraries related to GCC,
need to be in the same directory as the {\tt ld-linux.so}. To adhere
to the  restriction, the \pipglibc\ package's \pipcmd{piplnlibs} shell
script creates symbolic links of the shared objects in the {\tt
  /lib64} directory. 

\subsubsection{Loading Program by {\tt dlmopen}}

A PIE program cannot be loaded using the \linuxfunc{dlmopen} function
in CentOS/RedHat 8 (or maybe newer ones)\footnote{Refer
\url{https://sourceware.org/bugzilla/show_bug.cgi?id=11754\#c15}}. This
Glibc restriction is to be circumvented via the \pipcmd{pip-unpie}
tool. Instead of being called directly by users, this program is run
automatically by the \pipcmd{pipcc} or \pipcmd{pipfc} when a PiP
executable is created. 

\subsection{Glibc {\tt RPATH} Setting}

Every program, including PiP, has its {\tt RPATH}s added automatically
when using Spack\footnote{\url{https://spack.io}}. While using Spack
to install PiP-glibc, an issue occurs. Spack adds the {\tt RPATH}
value to the compiled \pipglibc\, but loading Glibc with the {\tt
  RPATH} setting is not allowed in CentOS/Redhat 8. \pipglibc\ has a
program (\pipcmd{annul_rpath}) to unset the {\tt RPATH} setting of the
\pipglibc\ that has been built in order to prevent this. 

\subsection{Linux}

\subsubsection{Heap Segment}\label{sec:heap}

Another problem exists, although it originates with the Linux kernel
rather than Glibc. Let's begin by discussing the structure of an
address space before attempting to clarify this problem.

\lstinputlisting[style=example, basicstyle=\tiny\tt, 
  caption={A Memory Map Example},
  label=out:maps] {glibc/examples/maps.out}

Listing~\ref{out:maps} is an example of the output of the command
"{\tt cat /proc/\option{PID}/maps}." Here, the command "{\tt pip-exec
  -n 2./a.out}" was run, producing one pip-exec (PiP root) process and
two ./a.out PiP tasks. The file, {\tt /proc/\option{PID}/maps},
typically lists every memory segment in the process 
PID's address space. A loaded shared object has three or four segments
that are sequentially executable, gap (if any), constants, and
data. The \linuxfunc{mmap}ed filename is shown in the rightmost column
of a line, and the memory segment's permission is shown in the second
column from the left. '{\tt r}' stands for readable, '{\tt w}' for
writable, '{\tt X}' for executable, and '{\tt p}' for private
(copy-on-write). The filenames of a few special segments, including
{\tt [stack]}, {\tt [heap]}, and others, are enclosed in a pair of
square brackets. As their names imply, these were developed by the
Linux kernel for certain purposes. The \linuxfunc{mmap} system call
may create a segment without a filename. 

Keep in mind that this is the address space for executing one PiP root
and two PiP tasks, giving access to all of the three tasks'
segments. Listing~\ref{out:maps} shows that there are three sets of a
shared library and only one set of {\tt ld-linux.so} ({\tt
  ld-2.28.so}), and that all three jobs have exactly the same content
in {\tt /proc/\option{PID}/maps}. 

Typically, there is just one heap segment per address space, which is
primarily used by \linuxfunc{malloc}. There is only one heap segment, as
demonstrated in this example, and three tasks share it.
By using the \linuxfunc{brk} system call, the size of the heap segment
can be increased or decreased. To create or deallocate heap memory,
\linuxfunc{brk} is often called twice, once to determine the current
heap end address and once to set the new heap end address. The
\linuxfunc{sbrk} function in Glibc does just this. Evidently, this API
is not at all thread-safe, hence PiP jobs cannot use the shared heap
memory in a secure manner.  

Thankfully, the \linuxfunc{malloc} functions in Glibc are made to
determine whether there are two or more name spaces and, if there are,
to utilize \linuxfunc{mmap} rather than the \linuxfunc{brk} system
call. As a result, PiP may use the Glibc \linuxfunc{malloc} functions
without any issues. Nevertheless, this shared heap may cause issues if
other routines use the \linuxfunc{brk} system call (or
\linuxfunc{sbrk} Glibc function), such as when replacing the 
Glibc \linuxfunc{malloc} routines with another \linuxfunc{malloc}
implementation. 

\subsubsection{Core File}

Imagine a catastrophe where all PiP tasks and their root processes
dump their core files on their own. Nowadays, a core file on Linux is
connected to a process (including threads inside of it). As a result,
the root and each PiP task may produce core, leading to the existence
of several core files. In this case, they all share the same address
space and the generated core files may vary.

I'll use an example to further illustrate. Let's say that PiP tasks $A$
and $B$ are operating in the same address space as the root $P$, and
that an error occurs, causing all three to output core files. The time it
takes to build each core file can vary slightly. When the first core
file of A, for example, is being produced, the other P and B are still
executing, and those ongoing processes may change the memory of the
shared address space. However, \linuxcmd{gdb} makes the assumption
that a core file is a reliable snapshot of the CPU and memory
state. The PiP scenario mentioned above disproves this notion. The
same thing may occur if $B$ produces another core file, which may or may
not be related to the fault on $A$. Hence, debugging from a core file is
not supported by the \pipcmd{pip-gdb} command
(Section~\ref{sec:pip-gdb}). A PiP-aware OS kernel with consistent
core files is required to address this problem.  

\subsection{Tools}

According to the description, PiP sets a unique set of the
\linuxfunc{clone} flags. This causes some tools to malfunction. Below
is a list of tools that, as of this writing, are known to work or
not\footnote{\linuxcmd{ltrace} depends on its version}.  

\begin{table}[ht]
  \centering
  \caption{Compatibility of Tools}
  \vspace{3mm}
  \begin{tabular}{c|c}
    \hline
    Compatible & Incompatible \\
    \hline
        {\tt strace} & {\tt valgrind} \\
        {\tt ltrace} \\
        \hline
  \end{tabular}
\end{table}
