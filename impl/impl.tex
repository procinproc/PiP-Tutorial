
\section{PiP Implementation}

\subsection{Spawning Tasks}\label{sec:spawn-details}

Before PiP version 2.4, PiP tasks were created with the procedure as
follows;

\begin{enumerate}
\item The spawned program is loaded by calling {\tt dlmopen()},
\item Glibc is initialized in the execution context of the loaded
  program,
\item Call {\tt clone()} or {\tt pthread_create()} (chosen by the
  \pipterm{PIP_MODE} setting) to spawn the PiP task,
\item The before hook is called if any, and finally
\item Jump into the starting function.
\end{enumerate}

From PiP version 2.4, the wrapper functions listed in
Table~\ref{tbl:pip-wrapper} were introduced. When implementing the
wrapper functions, I noticed that wrapping the {\tt dlsym()} is
almost impossible.

A function wrapper is usually implemented as; 1) obtain the wrapping
function address by calling the {\tt dlsym()} with the {\tt RTLD_NEXT}
argument, 2) do the wrapping job before and/or after calling the
original function. The most of the Glibc {\tt malloc} routines has
the other weak symbols ({\tt malloc()} and {\tt __libc_malloc()}, for
example) and users can call the Glibc {\tt malloc} routines without
calling {\tt dlsym()}. If there is no such weak symbol, we cannot
create a wrapper function for {\tt dlsym()}. How can I
wrap a Glibc function without calling {\tt dlsym()}?

To solve this issue, I implemented another program, so called
\pipterm{ldpip.so} to load the PiP library and user program. here is
the details of new spawning process;

\begin{enumerate}
\item Load \pipterm{ldpip.so} in the PiP library package by calling
  {\tt dlmopen()} and jump into a function defined inside of it,
\item The starting function of \pipterm{ldpip.so} initializes Glibc,
\item Obtain Glibc function addresses to wrap them later by
  \pipterm{libpip.so},
\item Load \pipterm{libpip.so} by calling {\tt dlopen()},
\item Load a user program by calling {\tt dlopen()},
\item Call {\tt clone()} or {\tt pthread_create()} (chosen by the
  \pipterm{PIP_MODE} setting) to spawn the PiP task,
\item Jump into a function inside of PiP library and initialize the
  PiP library, 
\item The before hook is called if any, and finally
\item Jump into the starting function in the user program.
\end{enumerate}

At the time of loading \pipterm{ldpip.so}, no wrapper functions are
defined in this program and obtaining the Glibc function addresses is
easy, just referencing them. After loading the \pipterm{libpip.so} and
jumping into a function defined in \pipterm{libpip.so} where the
wrapping functions are defined, the Glibc functions to be wrapped are
now wrapped by using the function table created by
\pipterm{ldpip.so}\footnote{If actual dynamic linking would be done in
the order of {\tt dl[m]open()}, then the wrapping functions in {\tt 
  libpip.so} would not work as described here. As long as I checked,
the Glibc ({\tt libc.so}) is at the last of the search order of {\tt
  ld-linux.so}, and this works.}.

The Glibc initialization\footnote{Calling {\tt __ctype_init()}} must
be done with the execution context (Section~\ref{sec:context}) of the
  spawned PiP task. In the older version of PiP library, this was done
  by; 1) calling {\tt dlsym()} to the loaded handle, returned by {\tt
    dl[m]open()}, to obtain the initialization function
and then 2) call the function. In the new implementation, the
initialization was done by simply calling the initialization function
from the \pipterm{ldpip.so} where the execution context is the same
with that of PiP task. 

Thus, by introducing PiP loader program (\pipterm{ldpip.so}), things
can go in a simpler way. 


\subsection{Calling {\tt clone()} System Call}\label{sec:clone}

\subsection{Name of PiP Tasks}\label{sec:proc-name}




