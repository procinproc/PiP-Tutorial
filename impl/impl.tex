
\section{PiP Implementation and design choices}

\subsection{Spawning Tasks}\label{sec:spawn-details}

Prior to PiP version 2.4, the following steps were used to launch PiP
tasks:

\begin{enumerate}
\item The spawning program is loaded by calling \linuxfunc{dlmopen},
\item Glibc is initialized in the execution context of the loaded
  program,
\item Call \linuxfunc{clone} or \linuxfunc{pthread_create} (chosen
  by the \pipenv{PIP_MODE} environment setting) to spawn the PiP task,
\item The before hook is called if any, and finally
\item Jump into the starting function.
\end{enumerate}

Wrapper functions for the Glibc functions listed in
Table~\ref{tbl:pip-wrapper} were first introduced in PiP version
2.4. When developing these wrapper functions, I discovered that it is
nearly impossible to wrap the \linuxfunc{dlsym}.
A function wrapper is usually implemented as; 1) obtain the wrapping
function address by calling the \linuxfunc{dlsym} with the
\linuxdef{RTLD_NEXT} 
argument, 2) do the wrapping job before and/or after calling the
original function. The most of the Glibc \linuxfunc{malloc} routines has
the other weak symbols (\linuxfunc{malloc} and
\linuxfunc{__libc_malloc}, for example) and users can call the Glibc
\linuxfunc{malloc} routines without  
calling \linuxfunc{dlsym}. If there is no such weak symbol, we cannot
create a wrapper function for \linuxfunc{dlsym}. How can I
wrap a Glibc function without calling \linuxfunc{dlsym}?

I used another program, dubbed \pipterm{ldpip.so}, to resolve this
problem. thus to load the PiP library and user application. the method
for new spawning is described here;

\begin{enumerate}
\item Load \pipterm{ldpip.so} attached in the PiP library package by
  calling \linuxfunc{dlmopen} and jump into a function defined inside
  of it, 
\item The starting function of \pipterm{ldpip.so} initializes Glibc,
\item Obtain Glibc function addresses to wrap them later by the PiP
  library (\pipterm{libpip.so}),
\item Load \pipterm{libpip.so} by calling \linuxfunc{dlopen},
\item Load a user program by calling \linuxfunc{dlopen},
\item Call \linuxfunc{clone} or \linuxfunc{pthread_create} (chosen
  by the \pipenv{PIP_MODE} environment setting) to spawn the PiP task, 
\item Jump into a function inside of PiP library and initialize the
  PiP library, 
\item The before hook is called if any, and finally
\item Jump into the starting function in the user program.
\end{enumerate}

No wrapper functions are defined at the moment \pipterm{ldpip.so} is
loaded, making it simple to access the Glibc function addresses by
simply referencing them. The Glibc functions that need to be wrapped
are now wrapped using the function address table built by
\pipterm{ldpip.so} after loading \pipterm{libpip.so}.

It is necessary to run \linuxfunc{__ctype_init} with the execution
context (Section~\ref{sec:context}) of the launched PiP task in order
to initialize Glibc. In an earlier version of the PiP library, this
was accomplished by first obtaining the initialization function by
calling \linuxfunc{dlsym} on the loaded handle returned by
\linuxfunc{dlopen}, and then by executing the 
initialization function itself. The initialization process in the new
implementation consisted of calling the initialization method directly
from the ldpip.so file, where the execution context was the same as
that of the PiP task. So, things can go in an easier manner by
introducing PiP loader program (\pipterm{ldpip.so}).

\subsection{Calling {\tt clone()} System Call}\label{sec:clone}

PiP library uses a unique flag combination to call the
\linuxfunc{clone} system call, as explained in
Section~\ref{sec:spawn-details}. Since the \linuxfunc{clone} system
call takes a lot of inputs, some of which are difficult to implement,
I chose to wrap \linuxfunc{pthread_create} and \linuxfunc{clone} in
order to change only the flag setting.   

Wrapping the \linuxfunc{clone} system call is a challenge because it
is used by many libraries in addition to the PiP library (e.g.,
PThread library). When the PiP library calls the function, the flags
must be modified; however, when some other libraries call the same
method, the flags should not be changed. This is a circumstance that
cannot be handled by a straightforward function wrapping.

The test-and-set atomic instruction was used to construct a specific
locking mechanism in order to address this problem and prevent the
\linuxfunc{clone} system call from being invoked simultaneously from
the PiP library and from another source. The {\tt test-and-set}
instruction is used by the PiP library to lock when it is going to run
\linuxfunc{pthread_create} with the value of current thread ID (TID),
which ultimately invokes the \linuxfunc{clone} system call. The wrapper 
function of the \linuxfunc{clone} firstly tries to lock, but it fails
with value of the current TID, then it is the case of calling from the
PiP library. If the lock is successful, a different library will call
it. The original \linuxfunc{clone} is called with the altered flags in
the first scenario. In the latter scenario, the wrapper function call
calls \linuxfunc{clone} with the same input. Naturally, the lock is
released after the original \linuxfunc{clone} system call returns. 

\subsection{Execution Mode in Details}

The PiP's \pipterm{process mode} has two sub-modes:
\pipterm{process:preload} and \pipterm{process:pipclone}\footnote{In
  PiP implementation earlier than version 2.4, there was another mode 
\pipterm{process:got}. But this becomes obsolete in the newer
versions.}. Wrapping the \linuxfunc{clone} function mentioned above 
implements the \pipterm{process:preload} mode. The
\pipterm{process:pipclone} mode implements another
\linuxfunc{pthread_create}-like function in the \pipglibc\ (Section
3.2.2). If the PiP library is configured to use the \pipglibc,
\pipterm{process:pipclone} is taken; otherwise,
\pipterm{process:preload} is taken.  

\subsection{Name of PiP Tasks}\label{sec:proc-name}

The ability of the \pipcmd{pips} command (Section~\ref{sec:pips}) to
distinguish between PiP tasks and other customary processes and/or
threads may be unclear to certain readers. Each process and thread in
Linux has a name, which is visible in the COMMAND column of the top
command. The \linuxfunc{prctl} system call (in \pipterm{process mode})
or the \linuxfunc{pthread_ setname np} call (in \pipterm{pthread
  mode}) are used by the PiP library to set the command name. The 
first two characters of the name are used by the PiP library
(Table~\ref{tbl:name-1} and \ref{tbl:name-2})). 

\begin{table}[ht]
  \centering
  \caption{Command Name Setting (1st char.)}\label{tbl:name-1}
  \vspace{3mm}
  \begin{tabular}{c|c|l}
    \hline
    First char. & Distinction & \multicolumn{1}{c}{Note} \\
    \hline
    {\tt R} & PiP Root &  \\
    {\tt 0..9} & PiP Task & the least significant digit of \PIPID \\
    \hline
  \end{tabular}
\end{table}

\begin{table}[ht]
  \centering
  \caption{Command Name Setting (Second char.)}\label{tbl:name-2}
  \vspace{3mm}
  \begin{tabular}{c|lc|l}
    \hline
    2nd char. & Execution Mode & Abbreviation & Note\\
    \hline
    {\tt :} & \pipterm{process:preload} & {\tt L} & \\
    {\tt ;} & \pipterm{process:pipclone} & {\tt C} & \\
    {\tt .} & \pipterm{process:got} & {\tt G} & obsolete \\
    ${\tt \vert}$ & \pipterm{pthread} & {\tt T} & \\
    \hline
  \end{tabular}
\end{table}

PiP takes up the first two characters of the name, which can be up to
16 characters. The original command name is represented by the final
14 characters. The letters used by the pip-mode command
(Section~\ref{sec:pip-mode}) to specify the PiP execution mode are
shown in the abbreviation column of Table~\ref{tbl:name-2}. The
\pipcmd{pips} command may now identify the normal processes or threads
from the PiP roots and PiP tasks by using those first 2 characters of
the command name. 
