
\section{PiP Implementation}

\subsection{Spawning Tasks}\label{sec:spawn-details}

Before PiP version 2.4, PiP tasks were created with the procedure as
follows;

\begin{enumerate}
\item The spawned program is loaded by calling \linuxfunc{dlmopen()},
\item Glibc is initialized in the execution context of the loaded
  program,
\item Call \linuxfunc{clone()} or \linuxfunc{pthread_create()} (chosen
  by the \pipenv{PIP_MODE} environment setting) to spawn the PiP task,
\item The before hook is called if any, and finally
\item Jump into the starting function.
\end{enumerate}

From PiP version 2.4, the wrapper functions listed in
Table~\ref{tbl:pip-wrapper} were introduced. When implementing the
wrapper functions, I noticed that wrapping the \linuxfunc{dlsym()} is
almost impossible.

A function wrapper is usually implemented as; 1) obtain the wrapping
function address by calling the \linuxfunc{dlsym()} with the
\linuxdef{RTLD_NEXT} 
argument, 2) do the wrapping job before and/or after calling the
original function. The most of the Glibc \linuxfunc{malloc} routines has
the other weak symbols (\linuxfunc{malloc()} and
\linuxfunc{__libc_malloc()}, for example) and users can call the Glibc
\linuxfunc{malloc} routines without  
calling \linuxfunc{dlsym()}. If there is no such weak symbol, we cannot
create a wrapper function for \linuxfunc{dlsym()}. How can I
wrap a Glibc function without calling \linuxfunc{dlsym()}?

To solve this issue, I implemented another program, so called
\pipterm{ldpip.so} to load the PiP library and user program. here is
the details of new spawning process;

\begin{enumerate}
\item Load \pipterm{ldpip.so} in the PiP library package by calling
  \linuxfunc{dlmopen()} and jump into a function defined inside of it,
\item The starting function of \pipterm{ldpip.so} initializes Glibc,
\item Obtain Glibc function addresses to wrap them later by
  \pipterm{libpip.so},
\item Load \pipterm{libpip.so} by calling \linuxfunc{dlopen()},
\item Load a user program by calling \linuxfunc{dlopen()},
\item Call \linuxfunc{clone()} or \linuxfunc{pthread_create()} (chosen
  by the \pipenv{PIP_MODE} environment setting) to spawn the PiP task, 
\item Jump into a function inside of PiP library and initialize the
  PiP library, 
\item The before hook is called if any, and finally
\item Jump into the starting function in the user program.
\end{enumerate}

At the time of loading \pipterm{ldpip.so}, no wrapper functions are
defined in this program and obtaining the Glibc function addresses is
easy, just referencing them. After loading the \pipterm{libpip.so} and
jumping into a function defined in \pipterm{libpip.so} where the
wrapping functions are defined, the Glibc functions to be wrapped are
now wrapped by using the function table created by
\pipterm{ldpip.so}\footnote{If actual dynamic linking would be done in
the order of \linuxfunc{dl[m]open()}, then the wrapping functions in {\tt 
  libpip.so} would not work as described here. As long as I checked,
the Glibc ({\tt libc.so}) is at the last of the search order of {\tt
  ld-linux.so}, and this works.}.

The Glibc initialization\footnote{Calling \linuxfunc{__ctype_init()}} must
be done with the execution context (Section~\ref{sec:context}) of the
  spawned PiP task. In the older version of PiP library, this was done
  by; 1) calling \linuxfunc{dlsym()} to the loaded handle, returned by
  \linuxfunc{dl[m]open()}, to obtain the initialization function 
and then 2) call the function. In the new implementation, the
initialization was done by simply calling the initialization function
from the \pipterm{ldpip.so} where the execution context is the same
with that of PiP task. 

Thus, by introducing PiP loader program (\pipterm{ldpip.so}), things
can go in a simpler way. 


\subsection{Calling {\tt clone()} System Call}\label{sec:clone}

As described in Section~ref{sec:spawn-details}, PiP library calls the
\linuxfunc{clone()} system call with a special flag combination. The
\linuxfunc{clone()} system call has many arguments and some of them
are hard to implement, I decided to wrap the \linuxfunc{clone()}
system call to modify only the flag setting.

One issue to wrap the \linuxfunc{clone()} system call is that the
\linuxfunc{clone()} is called not only the PiP library, but also some
other libraries (e.g., PThread librray). A simple function wrapping
cannot handle both situations where the flags needs to be changed when
it is called by the PiP library but it should not change the flag when
called by some other libraries.

To solve this issue and protect the \linuxfunc{clone()} system call from
being called from PiP library and from another simultaneously, a
special locking mechanism was implemented by using the
{\tt\term{test-and-set}} atomic instruction. When the PiP library is about
to call \linuxfunc{pthread_create()} call, which eventually calls the
\linuxfunc{clone()} system call, it locks by using the {\tt\term{test-and-set}}
instruction with the value of current thread ID (TID). The wrapper
function of the {\tt clone} firstly tries to lock, but it fails with
value of the current TID, then it is the case of calling from the PiP
library. If the lock succeeds, then it is called by some other
library. In the former case, the original \linuxfunc{clone()} is called with
the modified flags. In the latter case, the \linuxfunc{clone()} is called
with the same argument with the wrapper function call. Needless to say,
the lock is unlocked after returning from the original \linuxfunc{clone()}
system call.

\subsection{Execution Mode in Details}

There are two sub-modes in the PiP's \pipterm{process mode},
\pipterm{process:preload} and \pipterm{process:pipclone}\footnote{In
PiP implementation earlier than version 2.4, there was another mode
\pipterm{process:got}. But this becomes obsolete in the newer
versions.}. The \pipterm{process:preload} mode is implemented by
wrapping the \linuxfunc{clone()} function described above and the
\pipterm{process:pipclone} 
is implemented to have another \linuxfunc{pthread_create()} like function
implemented in the patched-glibc (Section~\ref{sec:name-space}). If
  PiP library is configured to use the patched-glibc, then
  \pipterm{process:pipclone} is taken, otherwise \pipterm{process:preload}
  is taken. 

\subsection{Name of PiP Tasks}\label{sec:proc-name}

Some readers may wonder how the \pipcmd{pips} command
(Section~\ref{sec:pips}) can distinguish the PiP tasks and the other
normal processes and/or threads. In Linux, each process and thread can
have a name, which can be seen by the \linuxcomm{top} command in the {\tt
  COMMAND} column. The PiP library sets the command name by calling
the \linuxfunc{prctl()} system call (in \pipterm{process mode} or
\linuxfunc{pthread_setname_np()} call (in \pipterm{pthread mode}. The
PiP library uses the first two characters of the name
(Table~\ref{tbl:name-1} and \ref{tbl:name-2})).

\begin{table}[ht]
  \centering
  \caption{Command Name Setting (1st char.)}\label{tbl:name-1}
  \vspace{3mm}
  \begin{tabular}{c|c|l}
    \hline
    First char. & Distinction & \multicolumn{1}{c}{Note} \\
    \hline
    {\tt R} & PiP Root &  \\
    {\tt 0..9} & PiP Task & the least significant digit of \PIPID \\
    \hline
  \end{tabular}
\end{table}

\begin{table}[ht]
  \centering
  \caption{Command Name Setting (Second char.)}\label{tbl:name-2}
  \vspace{3mm}
  \begin{tabular}{c|lc|l}
    \hline
    2nd char. & Execution Mode & Abbreviation & Note\\
    \hline
    {\tt :} & \pipterm{process:preload} & {\tt L} & \\
    {\tt ;} & \pipterm{process:pipclone} & {\tt C} & \\
    {\tt .} & \pipterm{process:got} & {\tt G} & obsolete \\
    ${\tt \vert}$ & \pipterm{pthread} & {\tt T} & \\
    \hline
  \end{tabular}
\end{table}

The name may have up to 16 characters and PiP occupies the first two
characters The remaining 14 characters are used for representing
original command name. The abbreviation column in
Table~\ref{tbl:name-2} shows the characters used by the
\pipcmd{pip-mode} command (Section~\ref{sec:pip-mode}) to specify the
PiP execution mode.
The \pipcmd{pips} command can now distinguish the normal processes or
threads from the PiP roots and PiP tasks by using those first 2
characters of the command.
