
\section{Remaining Issues}

\subsection{Retrieving Memory}\label{sec:retrive}

Let us suppose a case where PiP task $A$ pass a pointer to PiP task
$B$ (Listing~\ref{out:export-import}, for example). After then, task
$A$ terminates for some reason. What if task $B$ tries to dereference
the pointer to access data which task $A$ had? This situation can also
happen if the string obtained by calling {\tt getenv()} is passed to
the other task. The consequence of this may introduce difficult
situation hard to debug. This situation must be detected by compilers
and/or tools which are aware of PiP-style execution model.

So, I decided not to reclaim any memory resources when a task
terminates, not calling {\tt dlclose()} nor {\tt free()}. In the
current PiP implementation, \PIPID\ can be allocated only once. And
not releasing memory resource will not cause further problem.
