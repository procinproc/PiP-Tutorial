
\section{Remaining Issues}

\subsection{Retrieving Memory}\label{sec:retrive}

Assume that PiP task $A$ passes a pointer to PiP task
$B$. (Listing~\ref{out:export-import}, 
for example). Then task $A$ exits for a reason. What if task $B$ tries to
access data that task A has by dereferenceing the pointer? This
scenario may also occur if the other job receives the string obtained
by calling \linuxfunc{getenv}. The result could result in challenging
circumstances that are challenging to troubleshoot. Compilers and/or
tools that are familiar with the execution model used by PiP must be
capable of identifying this situation.

I therefore made the decision to not call \linuxfunc{dlclose} or free
any memory resources when a task terminates. PIPID can only be allocated once
in the PiP implementation as it stands right now. And there won't be
any more issues if memory resources are not released.
