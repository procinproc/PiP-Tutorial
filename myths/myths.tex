
\section{Myths on PiP}

\subsection*{I cannot see the difference between shared memory and
  what PiP does}

The shared memory model enables to access the data owned by the other
process. While the POSIX share memory model allows to share only newly
allocated memory region, while another shared memory mode provided by 
XPMEM\footnote{\url{https://github.com/hpc/xpmem}} allows for the
other process to access any memory region. Thus both look
the same in terms of accessing the data owned by the other.

However, the mechanisms of both memory models are quite different. To
have a shared memory, one must call a system call to ask OS kernel to
have the shared memory. This kind of system calls, modifying the
memory mapping, are quite expensive. On the other hand, PiP tasks are 
mapped in one memory address, and a PiP task can access any data owned
by the others once the addresses of the data are known, without calling
any expensive system calls. In terms of how memory regions are mapped,
I name the way what PiP does \pipterm{shared address space model}, as
oppose to the \term{shared memory model}. 

If the data to be shared are scattered in an address space and hard to
pack in a memory region, or the shared data are being dynamically
allocated, then the \term{shared address space model} has the
advantage.  

\subsubsection*{Sharing an address with multiple programs can be a
  severe security issue}\label{sec:security-issue}

PiP allows to run programs sharing the same address space. The
most important point here is to make information exchange among
programs easy and efficient. If there is no information exchange among
them, there is no reason to run them with the PiP environment.

Basically, communicating programs share the same fate. Even a most
simple case where two programs are connected by using the Linux/Unix
pipe, one of the programs dies, the other programs also dies by
receiving the \linuxdef{SIGPIPE} signal. Communicating programs agree with
others when to communicate and how to communicate. The PiP case is no
exception.

\subsubsection*{Sharing address space makes debugging difficult}

It is true if one of the processes in a PiP environment destroy the
data owned by the other(s) may lead to a catastrophic result. If this
is done maliciously, then this cannot be avoided (see also
above). If the destruction is triggered by a
software bug, then this might be harder-to-debug than that of
multi-process model.  There are two points here; 1) the higher 
possibility of destructing of actual data, not accessing invalid memory
region (\linuxdef{SIGSEGV}), and 2) there are multiple execution entities.

The \term{ASLR} can be some help for the former point. If \term{ASLR}
is enabled, then the phenomenons of the bug can vary time to time. The
situation of the latter point is almost the same with the multi-thread case.

Anyway, I have no experiences for having bugs based on this situation
up until now. 

\subsubsection*{My program does not have any static variables and I do
  not need PiP.}

You may write programs without having any static variables. However,
the functions implemented in Glibc have many static variables. Your
runtime system may use some of the Glibc functions. So, in general, it
is very hard to write programs not having any static variables.

\subsubsection*{PiP may consume more memory than the other execution
  models}

The answer is yes, but not that much. Listing~\ref{out:maps} shows
how memory segments of running PiP tasks are mapped in one address
space. For example, Glibc ({\tt libc-2.28.so}),
Listing~\ref{out:glibc-segs} shows only the memory segments related to
Glibc, loaded three (3) times onto memory in this case, running one
PiP root and two PiP tasks each of which requires Glibc. Each segment
set is a memory map of {\tt libc-2.28.so}. All three segment sets are 
mapped from the same file, and the amount of consumed memory is the
same with having only one set. 

In the multi-process model, Each address space of a process has only
one {\tt libs-2.28.so} segment set, but another process has also the
same memory mapping of Glibc. Thus, roughly speaking, the amount of
memory required to run PiP tasks is almost the same with the one of
running multiple processes. However, in the multi-thread model, there
is only one variable segment shared among threads, regardless to the
number of threads. And the amount of memory for running PiP tasks is
larger than that of running multiple threads. 

\subsubsection*{There must be some hidden overhead for running PiP
  programs}

So far, it is know that there is one overhead which is larger than
the multi-process model. It is address space modification system
calls, such as \linuxfunc{mmap()} and \linuxfunc{brk()}. This is
because any modification of an address space must be locked inside of
the OS kernel and this lock contention results in larger
overhead. This situation is the same with the multi-thread model and
the overhead of \linuxfunc{mmap()} is larger than the multi-process
model but almost the same with the multi-thread model. There is no
other known additional overhead in PiP so far. 
