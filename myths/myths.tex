
\section{Some Myths on PiP}

\subsubsection*{Sharing an address with multiple programs can be a
  severe security issue}\label{sec:security-issue}

PiP allows to run programs sharing the same address space. The
most important point here is to make information exchange among
programs easy and efficient. If there is no information exchange among
them, there is no reason to run them with the PiP environment.

Basically, communicating programs share the same fate. Even a most
simple case where two programs are connected by using the Linux/Unix
pipe, one of the programs dies, the other programs also dies by
receiving the {\tt SIGPIPE} signal. Communicating programs agree with
others when to communicate and how to communicate. The PiP case is no
exception.

\subsubsection*{Sharing address space makes debugging difficult}

It is true if one of the processes in a PiP environment destroy the
data owned by the other(s) may lead to a catastrophic result. If this
is done maliciously, then this cannot be avoided (see also
above). If the destruction is triggered by a
software bug, then this might be harder-to-debug than that of
multi-process model.  There are two points here; 1) the higher 
possibility of destructing of actual data, not accessing invalid memory
region ({\tt SIGSEGV}), and 2) there are multiple execution entities.

The ASLR can be some help for the former point. If ASLR is enabled,
then the phenomenons of the bug can vary time to time. The situation
of the latter point is almost the same with the multi-thread case.

Anyway, I have no experiences for having bugs based on this situation
up until now. 

\subsubsection*{My program does not have any static variables and I do
  not need PiP.}

You may write programs without having any static variables. However,
the functions implemented in Glibc have many static variables. Your
runtime system may use some of the Glibc functions. So, in general, it
is very hard to write programs not having any static variables.

\subsubsection*{There must be some hidden overhead for running PiP
  programs}

So far, it is know that there is one overhead which is larger than
the multi-process model. It is address space modification system
calls, such as {\tt mmap()} and {\tt brk()}. This is because any
modification of an address space must be locked inside of the OS
kernel and this lock contention results in larger overhead. This
situation is the same with the multi-thread model and the overhead of
{\tt mmap()} is larger than the multi-process model but almost the
same with the multi-thread model. There is no other known additional
overhead in PiP so far. 
