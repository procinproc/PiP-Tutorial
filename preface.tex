
\section*{Motivation}

I stopped working in April 2022. Since then, I've written this PiP
tutorial while also maintaining the PiP library (which includes this
document; see \url{https://github.com/procinproc}).

PiP offers a distinctive and novel execution model. I've been having a
lot of problems developing the PiP library because Linux and the
current Glibc don't support the new execution mechanism. I intend for
the novel execution model that PiP offers to be significant in
computer science in the future, and I will do my best to keep the PiP
library up to date.

I am not confident in my ability to adapt PiP to the impending Glibc
and Linux because of the fact that Glibc and Linux change much more
frequently than PiP does due to manpower constraints. So I made the
decision to compose this document to share my thoughts.

This is the reason why this can serve as both an internal document and
a tutorial for utilizing PiP.

\section*{Expected Readers}

In this document, I made every effort to include as many sample
programs as I could. They are primarily written in C. A Linux version
of the latest PiP library has been tested (CentOS and Redhat, version
7 and 8). Therefore, readers must be conversant with Linux and C
programming. 

\section*{PiP Versions}

\begin{description}
\item[PiP Version 1]
  This is the very first release of PiP but it is obsolete now.
\item[PiP Version 2]
  This is the stable version of PiP.
\item[PiP Version 3]
  This is a beta version of PiP that uses User-Level Process (ULP) and
  Bi-Level Thread (BLT). There will be no description of BLT and
  ULP in this paper because they are not stable at the time of writing.
\end{description}

\section*{Other Documents}

Not all of the PiP library's features are covered in this
document. Consult the PiP reference manual (PDF file at
\url{https://github.com/procinproc/PiP/blob/pip-2/doc/latex-inuse/libpip-manpages.pdf}), 
an HTML document, or man pages (man pages and HTML documents will be
installed with PiP library) for this purpose.

\section*{Sample Programs}

Running the sample programs on a Docker environment running on Mac OSX
allowed for thorough testing of each program's functionality and the
production of all output examples. 

