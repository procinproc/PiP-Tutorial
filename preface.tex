
\section*{Motivation}

I retired from my work from April 2022. Since then, I wrote this PiP
tutorial while I am maintaining PiP library
(\url{https://github.com/procinproc}).

PiP provides a unique and new execution model. While I am maintaining
the PiP library, I have been facing many issues caused by the current
Glibc and Linux that are not aware of this new execution model. 
I hope the new execution model which PiP provides would play an
important role in computer science in the future, and I will continue
maintaining the PiP library until then.

Thinking the fact of that Glibc and Linux changing much more
frequently than PiP because of man-power, I am not confident with
adapting PiP with the upcoming Glibc and Linux. So, I decided to write
this document to leave my ideas.

This is the reason why this can not only be a tutorial for
using PiP, but also a internal document. 

\section*{Expected Readers}

I tried to have example programs in this document as much as
possible. Most of them are written in C. The current PiP library are
tested to run on Linux (CentOS and Redhat, version 7 and 8). So
readers must be familiar with C programming and Linux. 

\section*{Sample Programs}

All sample programs were tested to run and all the output examples are
obtained by running the sample programs on a Docker environment.

\section*{English}

To my shame, supposing many readers noticed already, my English is
quite poor. I would appreciate it if some of you would help me to
improve the readability of this document (and man page documents). If
this is the case, send me an e-mail ({\tt
  procinproc-info@googlegroups.com} or {\tt ahori@me.com}) and I will
give you the access right of the Github
(\url{https://github.com/procinproc}). 

