
\section{Rationale}\label{sec:rationale}

The steps to spawn a PiP task are as follows (a more complete
technique is provided in Section 3.1.1);

The steps to spawn a PiP task are: 1. construct a new name space by
using the Linux dlmopen()() function, 2. create a PiP task process (or
thread) by using the Linux clone() system call, and 3. enter the
starting function of a user application.


\begin{enumerate}
\item build a new name space by
using the Linux \linuxfunc{dlmopen} function,
\item create a PiP task process (or
thread) by using the Linux \linuxfunc{clone} system call, and
\item enter the starting function of a user application.
\end{enumerate}

In contrast to \linuxfunc{dlopen}, the \linuxfunc{dlmopen} function
can construct a new name space. Here, the global symbol names
(functions and global variables) that must be resolved upon program
loading make up the name space. Functions and variables may be
privatized from the other PiP tasks by establishing a distinct name
space. 

It's crucial to call \linuxfunc{dlmopen} and \linuxfunc{clone} in the
correct order. As it looked pretty reasonable, at first I tried to
call them in the following order: clone(), followed by dlmopen(), but
this did not work at all. This is the rationale behind why only PiP
root can start PiP tasks and watch for their terminations. 

The loaded address of a program is fixed by default in some
circumstances (or in the majority of circumstances prior to
CentOS/Redhat 8). If so, PiP is unable to load more than one program
in the same address space. The PiP executables must be built as \PIE
(Position Independent Executable) files in order for the programs to
be able to be loaded at any arbitrary address. All PiP tasks must be
built as PIE, which means they must be built using the
\pipcmdopt2{pipcc}{pipfc}{piptask} option using \pipcmd{pipcc} or
\pipcmd{pipfc} (or nothing to use the default). Keep in mind 
that the PiP root program could not be PIE.

PiP task can be formed by running the loaded program in a new name
space with another thread. Unfortunately, nothing is ever that
easy. Many problems are caused by Glibc. These problems are discussed
in the following section.
