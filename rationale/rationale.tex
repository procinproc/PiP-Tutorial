
\section{Rationale}\label{sec:rationale}

The prosedure to spawn a PiP task is (more detailed procedure can be
found at Section~\ref{sec:spawn-details});

\begin{enumerate}
\item create a new name space by calling the Linux's {\tt dlmopen()}
  function,
\item create a PiP task process (or thread) by calling the Linux's
  {\tt clone()} system call, and 
\item jump into the starting function of a user program.
\end{enumerate}

The {\tt dlmopen()} function can create a new name space, unlike {\tt
  dlopen()}. Here, the {\it name space} is the global symbol names
(functions and global variables) to be resolved at loading a program. By
creating a new name space, functions and variables can be
privatized from the other PiP tasks. 

The order of calling the {\tt dlmopen()} and {\tt clone()} is very
important.  At first, I tried to call them in the order of calling
{\tt clone()} followed by {\tt dlmopen()}, because this way seemed to be
quite natural, however, this does not work at all. This the reason of
that only PiP root can spawn PiP tasks and wait for the terminations
of PiP tasks. 

In some cases (or, in most cases before CentOS/Redhat 8), the loaded
address of a program is fixed by default. If this is the case, PiP
cannot load multiple programs in the same address space. To enable
this, the PiP executables must be compiled as \PIE\ (Position
Independent Executable) so that the programs can be loaded at any
arbitrary address. All programs to be PiP tasks must be compiled as
\PIE, i.e., must be compiled with \pipterm{pipcc} or \pipterm{pipfc}
with the {\tt --piptask} option (or nothing to use the default). Note
that PiP root program may not be \PIE.  

By running the loaded program having a new name space with another
thread, PiP task can be created. Unfortunately, things are not that
simple. There are many issues coming from Glibc. The next section will
describe these issues. 
