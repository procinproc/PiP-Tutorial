
\section{Spawning Tasks - Advanced}\label{sec:spawn-adv}

Further features of \pipfunc{pip_spawn} and \pipfunc{pip_task_spawn}
that have not yet been explained will be covered in this section. The
function prototypes for these functions are included below for your
convenience;

\begin{lstlisting}[frame=tb]
int pip_task_spawn( pip_spawn_program_t *progp, [IN]
                    uint32_t coreno,            [IN]
		    uint32_t opts,              [IN]
		    int *pipidp,                [IN/OUT]
		    pip_spawn_hook_t *hookp );  [IN]

int pip_spawn( char *filename,                  [IN]
               char **argv,                     [IN]
               char **envv,                     [IN]
	       int coreno,                      [IN]
               int *pipidp,                     [IN/OUT]
	       pip_spawnhook_t before,          [IN]
               pip_spawnhook_t after,           [IN]
               void *hookarg );                 [IN]
\end{lstlisting}

In Section 1.2.1, the first argument of the \pipfunc{pip_task_spawn}
function is already covered in Section~\ref{sec:spawn-func}. To
compact the first three arguments of \pipfunc{pip_spawn}, use the
\pipstruct{pip_spawn_program_t} structure. 

\subsection{Start Function}

These PiP spawn routines eventually jump into the start function
({\main} or a user-specified one), as already demonstrated in
Listing~\ref{sec:spawn-func}. PiP requires knowledge of the start
function's address in order to enable this. Here, one of the two
prerequisites must be satisfied: 

\begin{itemize}
\item the starting function is defined as a global symbol, or
\item if the starting function is defined as a local symbol then the
  executable file must not be stripped.
\end{itemize}

The {\tt -rdynamic} option is used during program compilation with the
\pipcmd{pipcc} and \pipcmd{pipfc} to make the symbol global for the
{\main} function. When it comes to user-defined local symbols, the PiP
library reads the executable file to spawn and attempts to locate the
starting function using the ELF metadata. Unfortunately, if the local
symbol information is stripped, it is lost, and PiP is unable to
locate the starting function.

\subsection{Stack Size}

The \pipdef{PIP_STACKSIZE} environment variable can be used to control the
stack size of created PiP tasks. Its value can be prefixed with ``{\tt
  T},'' ``{\tt G},'' ``{\tt M},'' ``{\tt K},'' or ``{\tt B}'' which,
like the \linuxenv{OMP_STACKSIZE} environment 
provided by OpenMP, stands for the {\it TiB}, {\it GiB}, {\it MiB}, {\it
  KiB}, or {\it Byte} unit,
respectively. {\it KiB} is presummated in the absence of a suffix. The
environment variables \linuxenv{KMP_STACKSIZE},
\linuxenv{GOMP_STACKSIZE}, or \linuxenv{OMP_STACKSIZE} 
are similarly effective with the priority in this order unless
\pipenv{PIP_STACKSIZE} is provided. While \pipenv{PIP_STACKSIZE} only
impacts the stack size of PiP processes, \linuxenv{KMP_STACKSIZE},
\linuxenv{GOMP_STACKSIZE}, and \linuxenv{OMP_STACKSIZE}  
also affect the size of OpenMP threads.

\subsection{CPU Core Binding}

The spawned PiP task is bound to the designated CPU core by the {\tt
  coreno} argument. This is the $N$th core number by default. The
absolute core number should be {\tt OR}ed with the
\pipdef{PIP_CPUCORE_ABS} flag if users want to specify the absolute
core number. Specifying this flag might not have 
an impact on the core number specification if the core numbers are
continuous. Only certain CPU architectures with non-contiguous core
numbers exhibit this disparity (e.g., Fujitsu A64FX). To connect to
the same CPU cores as the root process executing the spawn function,
the {\tt coreno} argument might be \pipdef{PIP_CPUCORE_ASIS}.

\subsection{File Descriptors and Spawn Hooks}\label{sec:hooks}

Similar to how \linuxfunc{fork} works, file descriptors from the root
process are duplicated and provided to the newly spawned child in the
\pipterm{process mode}. File descriptors are simply shared by PiP root
and PiP tasks in the \pipterm{pthread mode}.

The file descriptor is closed after calling the \pipterm{before hook}
mentioned below (if any), and then the start function is called if the 
\term{close-on-exec} flag of a file descriptor owned by the root
process is set in process mode.

The structure containing the final three arguments of
\pipfunc{pip_spawn} is the final argument of
\pipfunc{pip_task_spawn}. Calling the \pipfunc{pip_spawn_hook}
function will set the \pipstruct{pip_spawn_hook_t} structure. Here is
the prototype;  

\begin{lstlisting}[frame=tb]
void pip_spawn_hook( pip_spawn_hook_t *hook,  [OUT]
		     pip_spawnhook_t before,  [IN]
		     pip_spawnhook_t after,   [IN]
		     void *hookarg ) {        [IN]
typedef int (*pip_spawnhook_t)( void* );
\end{lstlisting}

Before invoking the start method (such as \main), this structure's
{\tt before} function is called when a PiP task is created. And when
the PiP task is ready to finish, the {\tt after} function is
invoked. Both functions are invoked with the {\tt hookarg}-specified
argument, which can pass any kind of data.

In Linux/Unix, calling \linuxfunc{fork} and \linuxfunc{execve}
generally results in the creation of a new process. Here, the parent
process's file descriptors are transferred to the newly created child
process. In many instances, such file descriptors are closed or
duplicated, and other settings are changed in between calls to
\linuxfunc{fork} and \linuxfunc{execve}. The task is only 
created by one function in PiP, hence it is not possible to use the
same settings as when {\tt fork\&exec} is used. These hook functions are
offered in order to accomplish this. Here is an illustration of one of
these hook functions;

\lstinputlisting[style=program,
  caption={Before and After Hooks},
  label=prg:hook] {spawn-adv/examples/hook.c}

In this illustration, the \pipterm{before hook} copies the root
process' FD1 to FD10. Next, the job that was started uses FD10 to
write a message. The \pipterm{after hook} closes the FD10 finally.
You should be aware  that the root calls those hook functions to run
them. 

\lstinputlisting[style=example, 
  caption={Before and After Hooks - Execution}, label=out:hook]
                {spawn-adv/examples/hook.out}
