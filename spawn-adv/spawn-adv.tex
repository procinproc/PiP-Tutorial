
\section{Spawning Tasks - Advanced}\label{sec:spawn-adv}

In this section, other features, not described so far, of
\pipterm{pip_spawn()} and \pipterm{pip_task_spawn()} will be
explained. For convenience, the function prototypes of these functions
are shown below;

\begin{lstlisting}[frame=tb]
int pip_task_spawn( pip_spawn_program_t *progp, [IN]
                    uint32_t coreno,            [IN]
		    uint32_t opts,              [IN]
		    int *pipidp,                [IN/OUT]
		    pip_spawn_hook_t *hookp );  [IN]

int pip_spawn( char *filename,                  [IN]
               char **argv,                     [IN]
               char **envv,                     [IN]
	       int coreno,                      [IN]
               int *pipidp,                     [IN/OUT]
	       pip_spawnhook_t before,          [IN]
               pip_spawnhook_t after,           [IN]
               void *hookarg );                 [IN]
\end{lstlisting}

The first argument of \pipterm{pip_task_spawn()} function is already
described in Section~\ref{sec:spawn-func}. This is structure is to
pack the first three arguments of \pipterm{pip_spawn()}. 

\subsubsection{Start Function}

As already shown in Listing~\ref{sec:spawn-func}, these PiP spawn
functions eventually jumps into the start function ({\tt main()} or
user specified one). To enable this, PiP needs to know the address of
the start function. One of the following two conditions must be
met here;

\begin{itemize}
\item the starting function is defined as a global symbol.
\item if the starting function is defined as a local symbol then the
  executable file must not be stripped.
\end{itemize}

As for the {\tt main()} function, the \pipterm{pipcc} and
\pipterm{pipfc} compile programs with the {\tt -rdynamic} option to
make the symbol global. As for the user-defined local symbol, PiP
library read the executable file to spawn and tries to find the
starting function by using the ELF information. Unfortunately, the
local symbol information is lost if stripped, and PiP fails to find
the starting function.

\subsection{CPU Core Binding}

The {\tt coreno} argument is to bind the spawned PiP task to the
specified CPU core. By default, this is the $N$th core number. If
users want to specify the absolute core number, then the absolute core
number should be {\tt OR}ed with
\pipterm{PIP_CPUCORE_ABS}\footnote{This is because the core numbers
are not contiguous on some CPU architecture, i.e., Fujitsu A64FX CPU.}
Or, it can be \pipterm{PIP_CPUCORE_ASIS} to bind to the same CPU cores
as the root process calls the spawn function.

\subsection{File Descriptors and Spawn Hooks}

In the \pipterm{process mode}, file descriptors of the root process
are duplicated and passed to the spawned child in the same way of what
{\tt fork()} does. In the \pipterm{pthread mode}, files descriptors
are simply shared among PiP root and PiP tasks. 

If the close-on-exec flag of a file descriptor owned by the root
process is set in \pipterm{process mode}, then the file descriptor is
closed after calling the \pipterm{before hook} described above (if
any), and then jump into the start function.  

The last argument of \pipterm{pip_task_spawn()} is the structure
packing the last three arguments of \pipterm{pip_spawn()}. The
\pipterm{pip_spawn_hook_t} structure can be set by calling
\pipterm{pip_spawn_hook()} function. Here is the prototype;

\begin{lstlisting}[frame=tb]
void pip_spawn_hook( pip_spawn_hook_t *hook,  [OUT]
		     pip_spawnhook_t before,  [IN]
		     pip_spawnhook_t after,   [IN]
		     void *hookarg ) {        [IN]
typedef int (*pip_spawnhook_t)( void* );
\end{lstlisting}

The {\tt before} function in this structure is called when a
PiP task is created and before calling the start function (e.g.,
{\tt main()}). And the {\tt after} function is called when the PiP
task is about to terminate. Both functions are called with the argument
specified by the {\tt hookarg} to pass any arbitrary data. 

In general, a new process is created by calling {\tt fork()} and {\tt
execve()} in Linux/Unix. Here, file descriptors owned by
parent process are passed to the created child. In many cases, those
file descriptors are closed or duplicated and some other settings take
place between the calls of {\tt fork()} and {\tt execve()}. In PiP,
however, the task is created by only one function and there is no
chance to do the same settings with the ones of using {\tt
  fork\&exec}. These hook functions are provided for this purpose. Here 
is an example of these hook functions;  

\lstinputlisting[style=program,
  caption={Before and After Hooks},
  label=prg:hook] {spawn-adv/examples/hook.c}

In this example, FD1 of the root process is duplicated to FD10 by the
\pipterm{before hook}. Then the spawned task write a message via
FD10. Finally, the FD10 is closed by the \pipterm{after hook}
(Listing~\ref{out:const-dest}). 
Note that the execution of those hook functions are called by the root. 

\lstinputlisting[style=example, 
  caption={Before and After Hooks - Execution}, label=out:hook]
                {spawn-adv/examples/hook.out}

\subsection{PRELOAD}

{\tt LD_PRELOAD} does not work when spawning PiP tasks. This is
because {\tt dlmopen()} does not support this. Although it is
impossible to support the function of the {\tt LD_PRELOAD} without
modifying Glibc, PiP library supports similar function, but not
exactly the same one with  {\tt LD_PRELOAD}.

By setting \pipterm{PIP_PRELOAD} in the same way of {\tt
  LD_PRELOAD} (colon (:) separated list of shared objects), those
specified object files are loaded before loading user program. At the
time of loading objects in the \pipterm{PIP_PRELOAD}, the Glibc is
already loaded, and users cannot override the 
functions provided by Glibc.
