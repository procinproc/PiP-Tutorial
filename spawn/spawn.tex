
\section{Spawning PiP Tasks and Wating Terminations}

The \pipterm{pip-exec} command spawns PiP tasks. The
process which spawns PiP tasks is called {\bf (PiP) root} process. The
\pipterm{pip-exec} process is a PiP root process. 
PiP tasks spawned by the root process are mapped and executed in the
address space of the root. In this chapter, how to spawn PiP tasks
will be explained.

\subsection{Spawning PiP tasks}

\subsubsection{Spawning a program as PiP tasks}

Listing~\ref{prg:spawn-root} is an example of a PiP root program. It
spawns $N$ PiP tasks, where $N$ is specified by the first parameter
of the program. The \pipterm{pip_init()} function must be called to
initialize the PiP library before calling any other PiP functions,
although there some exceptions to this. The \pipterm{pip_init()} may
look strange because this function behaves differently depending
on if it is called from a PiP root or PiP task. The first argument is
output returning {\PIPID} of the calling task. The second input
arguments is to specify the maximum number of spawning PiP tasks. This
second argument becomes output if this is called by a PiP task,
returning the number specified by the root.
The \pipterm{pip_fin()} function works as the opposite of
\pipterm{pip_init()}, finalizing PiP library and freeing allocated 
resources. After calling \pipterm{pip_fin()}, most PiP library
functions return an error code ({\tt EPERM}.
\comment{The other arguments will be explained in the
  Section~\ref{sec:advanced}.} 

The \pipterm{pip_spawn()} function is called after then. The first and
second arguments are the same with the Linux's {\tt execve}
function; the first is to specify the executable file to be executed
and the second argument is to specify the parameters executing the
program. The third is to specify environment variables. When it is
{\NULL}, then value of the Glibc global variable {\tt environ} is
taken. Th fourth argument is to specify the CPU core number to bind
the spawned PiP task and which CPU core. In this example, the value of
\pipterm{PIP_CPUCORE_ASIS} means that the (CPU) core-bind should be
the same with the one when calling \pipterm{pip_spawn()}. The fifth
is an input and output argument and you can specify {\PIPID} or
set to \pipterm{PIP_PIPID_ANY} so that PiP library can choose
any. After calling \pipterm{pip_spawn()}, the argument returns the
actual {\PIPID}. 

\lstinputlisting[style=program, caption={Spawn ({\tt spawn-root})},
  label=prg:spawn-root] {spawn/examples/spawn-root.c}

Listing~\ref{prg:spawn-task} is very similar to the ``Hello World''
program in the previous section. The major difference here is calling
the \pipterm{pip_init()} function. Unlike root, this function call is
optional in the PiP task program. By calling this, you can get 
{\PIPID} and the number of maximum PiP tasks which are specified by the
root. Linsting~\ref{out:spawn} shows an example of the execution of
Listing~\ref{prg:spawn-root} and \ref{prg:spawn-task}. 

\lstinputlisting[style=program, caption={Spawn ({\tt spawn-task})},
  label=prg:spawn-task]
                {spawn/examples/spawn-task.c}

\lstinputlisting[style=example, 
  caption={Spawn - Execution}, label=out:spawn]
                {spawn/examples/spawn.out}

\subsubsection{Spawning myself}

A program can be both or either PiP root and PiP
task. Listing~\ref{prg:spawn-myself} shows an example of combining the
programs of Listing~\ref{prg:spawn-root} and \ref{prg:spawn-task}. We
hope you can understand the strange behavior of \pipterm{pip_init()}
function. The PiP root process also acts like a PiP task. It has a
special PIPID, {\tt PIP_PIPID_ROOT}. 
Listing~\ref{out:spawn-myself} shows the example of this
execution. 

\lstinputlisting[style=program, caption={Spawn Myself ({\tt spawn-myself})},
  label=prg:spawn-myself]
                {spawn/examples/spawn-myself.c}

\lstinputlisting[style=example, 
  caption={Spawn Myself - Execution}, label=out:spawn-myself]
                {spawn/examples/spawn-myself.out}

\subsubsection{Starting from other than {\tt main}}

PiP tasks start from the {\tt main()} function in the examples so
far. PiP allows for PiP tasks to start user-defined function other
than {\tt main()}. In this case, use the \pipterm{pip_task_spawn()}
function instead of calling the \pipterm{pip_spawn()} function.

\lstinputlisting[style=program, caption={Starting from user-defined
    function ({\tt userfunc})},
  label=prg:userfunc]
                {spawn/examples/userfunc.c}

Listing~\ref{prg:userfunc} is the program of this example. To decrease
the number of arguments to spawn a PiP task, the
\pipterm{pip_spawn_program_t} structure is defined. This structure
holds all information for spawning a program, including path to
executable file, function name, and so on. To hide the details of the
structure, \pipterm{pip_spawn_from_func()} function is also defined
to set these information. The user-defined function must have one
argument ({\tt void*}) and return an integer value which is the same
as the return value from the {\tt main()} function.

\lstinputlisting[style=example, 
  caption={Starting from user-defined function - Execution},
  label=out:userfunc] {spawn/examples/userfunc.out}

The \pipterm{pip_spawn()} was firstly introduced (from version
1). After then, I noticed users can start PiP tasks other than main,
and the \pipterm{pip_task_spawn()} function was
introduced (from version 2 or later). The
\pipterm{pip_spawn_program_t} structure must be set 
by calling the \pipterm{pip_spawn_from_main()} function when
starting from the {\tt main()} function. Listing~\ref{prg:mainfunc} is
the program rewritten version of Listing~\ref{prg:spawn-myself} by
using the \pipterm{pip_task_spawn()} and
\pipterm{pip_spawn_from_main()}.

\lstinputlisting[style=program, caption={Starting from main
    function ({\tt mainfunc})},
  label=prg:mainfunc]
                {spawn/examples/mainfunc.c}

\lstinputlisting[style=example, 
  caption={Starting from main function - Execution},
  label=out:mainfunc] {spawn/examples/mainfunc.out}


\subsection{Waiting for Terminations of PiP tasks}

As readers may have already noticed, the \pipterm{pip_wait()} is the
function to wait for terminations of the spawned PiP tasks. The
\pipterm{pip_wait()} function acts like the Linux's {\tt wait()}
function. In many cases, Linux's {\tt wait()} function works with PiP
tasks, but there is a certain case it does not. So, it is recommended
for users to use \pipterm{pip_wait()} function.

The argument of the \pipterm{pip_wait()} is the pointer to an integer
variable, the same with the Linux's {\tt wait()} call. The returned
integer can be examined by using the Linux's {\tt WIFEXITED}, {\tt
  WIFSIGNALED}, {\tt WEXITSTATUS}, {\tt WIFSIGNALED}, and {\tt
  WTERMISIG} macros.

\lstinputlisting[style=program, caption={Waiting for specified PiP task
    terminations ({\tt wait})}, label=prg:wait]
                {spawn/examples/wait.c}

\lstinputlisting[style=example, 
  caption={Waiting for specified PiP task terminations - Execution},
  label=out:wait] {spawn/examples/wait.out}

\pipterm{pip_wait()} waits for the PiP task termination specified by
        {\PIPID}. 
\pipterm{pip_wait_any()} function can wait for any PiP tasks and
        {\PIPID} and exit status are returned when terminated (See
        Listing~\ref{prg:waitany} and \ref{out:waitany}).
        \pipterm{pip_trywait()} and \pipterm{pip_trywait_any()} are
        the non-blocking versions of \pipterm{pip_wait()} and
        \pipterm{pip_wait_any()}, respectively. 

\lstinputlisting[style=program, caption={Waiting for any PiP task
    terminations ({\tt waitany})}, label=prg:waitany]
                {spawn/examples/waitany.c}

\lstinputlisting[style=example, 
  caption={Waiting for any PiP task terminations - Execution},
  label=out:waitany] {spawn/examples/waitany.out}


\subsection{Terminating PiP tasks}

PiP tasks and root can terminate their executions by calling
\pipterm{pip_exit()} function. This function acts like the Linux's
        {\tt exit()} function. As described above, it is recommended
        to use \pipterm{pip_exit()} 
        instead of {\tt exit()}, because the Linux's {\tt exit()}
        function works in most cases, however, there is a case it does
        not. Listing~\ref{prg:exit} and \ref{out:exit} show the
        example showing how \pipterm{pip_exit()} works.

\lstinputlisting[style=program, caption={PiP Task Termination
    function ({\tt exit})}, label=prg:exit] {spawn/examples/exit.c}

\lstinputlisting[style=example, 
  caption={PiP Task Termination - Execution},
  label=out:exit] {spawn/examples/exit.out}

\comment{
\subsection{Address map}

However, actual mapped addresses are up to Linux kernel. 
}
