
\section{Spawning PiP Tasks and Waiting Terminations}

The \pipcmd{pip-exec} command starts PiP tasks. (PiP) root process is
the name of the process that creates PiP tasks. The \pipcmd{pip-exec}
process is a root process.  
The root process's address space is used to map and execute PiP tasks
that it has spawned. This chapter will describe how to spawn PiP
tasks.

\subsection{Spawning PiP tasks}

\subsubsection{Spawning a program as PiP tasks}

Listing~\ref{prg:spawn-root} is an example of a PiP root program. It
spawns $N$ PiP tasks, where $N$ is specified by the first parameter
of the program. The \pipfunc{pip_init} function must be called to
initialize the PiP library before calling any other PiP functions,
although there some exceptions to this. The first argument is
output returning {\PIPID} of the calling task. In this case,
\pipdef{PIP_PIPID_ROOT} is returned, since the function is called by
the root. The second input arguments is to specify the maximum number
of spawning PiP tasks. The other arguments will be explained in 
Chapter~\ref{chap:advanced}.

The \pipfunc{pip_spawn} function is called after then. The first and
second arguments are the same with the Linux's \linuxfunc{execve}
function; the first is to specify the executable file to be executed
and the second argument is to specify the parameters executing the
program. The third is to specify environment variables. When it is
{\NULL}, then value of the Glibc global variable \linuxvar{environ} is
taken. Th fourth argument is to specify the CPU core number to bind
the spawned PiP task and which CPU core. In this example, the value of
\pipdef{PIP_CPUCORE_ASIS} means that the (CPU) core-bind should be
the same with the one when calling \pipfunc{pip_spawn}. The fifth
is an input and output argument, and you can specify a \PIPID\ or
set to the value of \pipdef{PIP_PIPID_ANY} so that PiP library can choose
any. After calling \pipfunc{pip_spawn}, the argument returns the
actual \PIPID.

The \pipfunc{pip_wait} is to wait the termination of the spawned
task. Its first argument is to specify the \PIPID\ of the terminating
task. The second parameter, although {\NULL} is set in this example,
is the same with the Linux's \linuxfunc{wait} function, returning
the terminating status of a task.

The \pipfunc{pip_fin} function works as the opposite of
\pipfunc{pip_init}, finalizing PiP library and freeing allocated 
resources. After calling \pipfunc{pip_fin}, most PiP library
functions return an error code (\linuxdef{EPERM}).

\lstinputlisting[style=program, caption={Spawn ({\tt spawn-root})},
  label=prg:spawn-root] {spawn/examples/spawn-root.c}

Listing~\ref{prg:spawn-task} is very similar to the ``Hello World''
program in the previous section. The major difference here is calling
the \pipfunc{pip_init} function.
The \pipfunc{pip_init} may
look strange because this function behaves differently depending
on if it is called from a PiP root or PiP task.
Unlike root, this function call is
optional in the PiP task program. By calling this, you can get 
{\PIPID} and the number of maximum PiP tasks which are specified by the
root. Linsting~\ref{out:spawn} shows an example of the execution of
Listing~\ref{prg:spawn-root} and \ref{prg:spawn-task}. 

\lstinputlisting[style=program, caption={Spawn ({\tt spawn-task})},
  label=prg:spawn-task]
                {spawn/examples/spawn-task.c}

\lstinputlisting[style=example, 
  caption={Spawn - Execution}, label=out:spawn]
                {spawn/examples/spawn.out}

\subsubsection{Spawning myself}

A program can be either the PiP task or the PiP root. Combining the
programs from Listings~\ref{prg:spawn-myself} and \ref{prg:spawn-task}
is demonstrated in Listing~\ref{prg:spawn-myself} as 
an example. We hope you can comprehend \pipfunc{pip_init}'s peculiar
behavior. The PiP root process functions similarly to a PiP task. It
has a unique \PIPID\ called the
\pipdef{PIP_PIPID_ROOT}. Listing~\ref{out:spawn-myself} provides an 
illustration of this execution. 

\lstinputlisting[style=program, caption={Spawn Myself ({\tt spawn-myself})},
  label=prg:spawn-myself]
                {spawn/examples/spawn-myself.c}

\lstinputlisting[style=example, 
  caption={Spawn Myself - Execution}, label=out:spawn-myself]
                {spawn/examples/spawn-myself.out}

\subsubsection{Starting from other than {\tt main}}\label{sec:spawn-func}

In the preceding instances, PiP tasks start from the {\main}
function. PiP enables tasks to launch user-defined functions other
than the \main. Use the \pipfunc{pip_task_spawn} function in this
situation rather than executing \pipfunc{pip_spawn}. 

\lstinputlisting[style=program, caption={Starting from user-defined
    function ({\tt userfunc})},
  label=prg:userfunc] {spawn/examples/userfunc.c}

The software for this example is listed in
Listing~\ref{prg:userfunc}. The \pipstruct{pip_spawn_program_t}
structure is defined to minimize the number of arguments 
needed to spawn a PiP process. All necessary data for starting a
program, such as the function name and path to the executable file,
are stored in this structure. The \pipfunc{pip_spawn_from_func}
function is also defined to set this information in order to mask the
structure's specifics. The user-defined function must take a single
argument {\tt (void*)} and return an integer value that is identical
to the {\main} function's return value.

\lstinputlisting[style=example, 
  caption={Starting from user-defined function - Execution},
  label=out:userfunc] {spawn/examples/userfunc.out}

The \pipfunc{pip_spawn} was initially released (from version
1). After that, I found users could launch PiP tasks other than the
\main, and the \pipfunc{pip_task_spawn} function had been
introduced (from version 2 or later). When beginning from the 
{\main} function, the \pipstruct{pip_spawn_program_t} structure must
be set by calling the \pipfunc{pip_spawn_from_main} function.  
Listing~\ref{prg:mainfunc} is a program that updates
Listing~\ref{prg:spawn-myself} to use the \pipfunc{pip_task_spawn}
and \pipfunc{pip_spawn_from_main} functions.  

\lstinputlisting[style=program, caption={Starting from main
    function ({\tt mainfunc})},
  label=prg:mainfunc]
                {spawn/examples/mainfunc.c}

\lstinputlisting[style=example, 
  caption={Starting from main function - Execution},
  label=out:mainfunc] {spawn/examples/mainfunc.out}


\subsection{Waiting for Terminations of PiP tasks}

As readers may have already noted, the \pipfunc{pip_wait} function
is used to watch for PiP tasks that have been started to finish. The
\pipfunc{pip_wait} function functions similarly to the
\linuxfunc{wait} function in Linux. Linux's 
\linuxfunc{wait} function frequently cooperates with PiP jobs,
however there is one instance where it does not. Therefore, it is
advised that users use the \pipfunc{pip_wait} function.  Similar to
the wait() function in Linux, the argument of \pipfunc{pip_wait} is
a pointer to an integer variable. The Linux \linuxdef{WIFEXITED},
\linuxdef{WIFSIGNALED}, \linuxdef{WEXITSTATUS},
\linuxdef{WIFSIGNALED}, and \linuxdef{WTERMISIG} macros can be
used to inspect the returned integer.

\lstinputlisting[style=program, caption={Waiting for specified PiP task
    terminations ({\tt wait})}, label=prg:wait]
                {spawn/examples/wait.c}

\lstinputlisting[style=example, 
  caption={Waiting for specified PiP task terminations - Execution},
  label=out:wait] {spawn/examples/wait.out}

\pipfunc{pip_wait} waits for the PiP task termination specified by
        {\PIPID}. 
\pipfunc{pip_wait_any} function can wait for any PiP tasks and
        {\PIPID} and exit status are returned when terminated (See
        Listing~\ref{prg:waitany} and \ref{out:waitany}).
        \pipfunc{pip_trywait} and \pipfunc{pip_trywait_any} are
        the non-blocking versions of \pipfunc{pip_wait} and
        \pipfunc{pip_wait_any}, respectively. 

\lstinputlisting[style=program, caption={Waiting for any PiP task
    terminations ({\tt waitany})}, label=prg:waitany]
                {spawn/examples/waitany.c}

\lstinputlisting[style=example, 
  caption={Waiting for any PiP task terminations - Execution},
  label=out:waitany] {spawn/examples/waitany.out}


\subsection{Terminating PiP tasks}

By using the \pipfunc{pip_exit} function, PiP tasks and root can stop
running. Comparable to Linux's \linuxfunc{exit} function is this function. As
stated above, it is advised to use \pipfunc{pip_exit} rather than
\linuxfunc{exit} since, while the Linux \linuxfunc{exit} function
typically works, there are some instances where it does
not. Listing~\ref{prg:exit} and \ref{out:exit} provide a working 
example of \pipfunc{pip_exit}.

\lstinputlisting[style=program, caption={PiP Task Termination
    function ({\tt exit})}, label=prg:exit] {spawn/examples/exit.c}

\lstinputlisting[style=example, 
  caption={PiP Task Termination - Execution},
  label=out:exit] {spawn/examples/exit.out}
