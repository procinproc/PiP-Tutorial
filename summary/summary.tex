
\section{Summary}

\subsection*{PiP root and PiP task}

\begin{itemize}
\item PiP programs must be compiled with the \pipterm{pipcc} 
  (for C and C++) or \pipterm{pipfc} (for  FOrtran) command.
\item PiP programs can run as PiP tasks by using the
  \pipterm{pip-exec} command. 
\item PiP programs can run as non-PiP tasks by invoking them as normal
  programs.
\item Unlike the conventional multi-thread model (i.e. OpenMP), static
  variables in a PiP program are privatized and each PiP task has its
  own set of the static variables.
\item Unlike the conventional multi-process model (i.e. MPI), PiP
  tasks may share the same address space and PiP tasks can access data
  owned by the other PiP tasks. 
\end{itemize}

\subsection*{PiP API}

\begin{itemize}
\item Most PiP functions return error code defined in Linux.
\item Every PiP task has a unique {\PIPID} per address space.
\item PiP root must initialize PiP library by calling
  \pipterm{pip_init()}. While child PiP task may or may not call the
  initialization function.
\item PiP root can spawn PiP tasks by calling the
  \pipterm{pip\_spawn()} or \pipterm{pip_task_spawn()} function.
\item To obtain the address for accessing data of the other PiP tasks,
  use the \pipterm{pip_named_export()} and
  \pipterm{pip_named_import()} functions.  
\item The \pipterm{pip_named_export()} and \pipterm{pip_named_import()}
  can be used to synchronize tasks. \pipterm{pip_barrier_wait} can
  also be used for tasks to synchronize.
\item The \pipterm{pip_exit()} function terminates the calling PiP
  task and PiP root.
\item PiP root can wait for the termination of a spawned PiP task, by
  calling one of the \pipterm{pip_wait()} function family.
\end{itemize}

\subsection*{Commands}

