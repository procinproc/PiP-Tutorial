
\section{Summary}

\subsection*{PiP root and PiP task}

\begin{itemize}
\item PiP applications must be compiled using the pipfc or pipcc
  commands, respectively, for C, C++, and Fortran.
\item The \pipcmd{pip-exec} command allows PiP programs to be executed as PiP
  tasks.
\item PiP programs can be invoked as regular programs to execute as
  non-PiP tasks.
\item Static variables in a PiP application are privatized, meaning
  that each PiP task has its own set of the static variables, in
  contrast to the traditional multi-thread paradigm (i.e. OpenMP).
\item In contrast to the traditional multi-process approach
  (i.e. MPI), PiP tasks have access to each other's data and share
  the same address space.
\end{itemize}

\subsection*{PiP API}

\begin{itemize}
\item The majority of PiP functions return a Linux-defined error code.
\item Per address space, each PiP task has an unique {\PIPID}.
\item PiP root must call \pipfunc{pip_init} to initialize PiP library. The
  initialization function may or may not be called by child PiP tasks.
\item By using the \pipfunc{pip\_spawn} or
  \pipfunc{pip_task_spawn} functions, the PiP root can start new PiP
  tasks. 
\item Use the \pipfunc{pip_named_export} and
  \pipfunc{pip_named_import} methods to find the address for gaining
  access to the data of the other PiP tasks.
\item \pipfunc{pip_barrier_wait} is used for tasks to
  synchronize. Synchronization can also be accomplished using
  PThreads' synchronization functions. 
\item The invoking PiP process and PiP root are terminated by the
  \pipfunc{pip_exit} method. 
\item By using a member of the \pipfunc{pip_wait} function family, the PiP
  root can wait for a created PiP task to finish.
\end{itemize}
 
