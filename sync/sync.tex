
\section{Timing Synchronization among PiP Tasks}

This section will explain about the timing synchronization among PiP
tasks.

\subsection{Barrier Synchronization}

Currently, there is only one synchronization method is supported by
the PiP library, it is barrier synchronization. The API of PiP's
barrier synchronization is borrowed from the one found in the PThread
library. There are three functions in PiP,
\pipfunc{pip_barrier_init()}, \pipfunc{pip_barrier_wait()}, and
  \pipfunc{pip_barrier_fin()}, corresponding to
  \linuxfunc{pthread_barreir_init()}, \linuxfunc{pthread_barrier_wait()} and
  \linuxfunc{pthread_barrier_destroy()}, respectively. 

\lstinputlisting[style=program, caption={Barrier Synchronization
    ({\tt barrier})}, label=prg:barrier] {sync/examples/barrier.c}

In Listing~\ref{prg:barrier}, the \pipfunc{pip_init()} function is
given a new non-{\tt NULL} value to the third argument. This is
another form of exporting a pointer from the root to spawnd PiP
tasks. In this example, the address of the \pipstruct{pip_barrier_t}
static variable is passed to children so that the children can
synchronize by calling \pipfunc{pip_barrier_wait()}.

To clarify the effect of the barrier synchronization, the
synchronization takes place only when the second parameter of the
program execution is not given, and then the return values of
\pipfunc{pip_gettime()} are shown by PiP tasks. The
\pipfunc{pip_gettime()} returns the current value of
\linuxfunc{gettimeofday()} in double format with the unit of seconds. 

The example of running of this program is shown in
Listing~\ref{out:barrier}. In the first run, the barrier
synchronization does not take place and large variance can be seen on
the \linuxfunc{gettimeofday()} values. In the second run, where the
barrier synchronization takes place, and smaller variance can be seen.

\lstinputlisting[style=example,
  caption={Barrier Synchronization - Execution},
  label=out:barrier] {sync/examples/barrier.out}

\subsection{Using PThread Synchronization}

Users can utilize the synchronization functions on PiP tasks provided
by the PThread library. This is simply because PiP tasks share the
same address space, just like threads.

\subsection{\tt pthread_barrier}

The same barrier synchronization can also be implemented by using the
\linuxfunc{pthread_barrier} functions. 
Listing~\ref{prg:pthread-barrier} is the program simply replacing
\pipfunc{pip_barrier} functions with the \linuxfunc{pthread_barrier}
functions. 

\lstinputlisting[style=program,
  caption={Pthread Barrier ({\tt
      pthread-barrier})},label=prg:pthread-barrier]
                {sync/examples/pthread-barrier.c}

\lstinputlisting[style=example, 
  caption={Pthread Barrier - Execution}, label=out:pthread-barrier]
                {sync/examples/pthread-barrier.out}
                
\subsection{\tt pthread_mutex}

Similarly, \linuxfunc{pthread_mutex} also works with PiP. 

\lstinputlisting[style=program,
  caption={Pthread Mutex ({\tt
      pthread-mutex})},label=prg:pthread-mutex]
                {sync/examples/pthread-mutex.c}

\lstinputlisting[style=example, 
  caption={Pthread Mutex - Execution}, label=out:pthread-mutex]
                {sync/examples/pthread-mutex.out}
