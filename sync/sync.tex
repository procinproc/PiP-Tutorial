
\section{Timing Synchronization among PiP Tasks}

This section will go over how PiP tasks' timing synchronization works.

\subsection{Barrier Synchronization}

The PiP library only supports barrier synchronization as a
synchronization mechanism at this time. Barrier synchronization in PiP
uses an API that was taken from the PThread library. PiP has three
functions that are equivalent to \linuxfunc{pthread_barrier_init()},
\linuxfunc{pthread_barrier_wait()}, and
\linuxfunc{pthread_barrier_destroy()}, respectively:
\pipfunc{pip_barrier_init()}, \pipfunc{pip_barrier_wait()}, and
\pipfunc{pip_barrier_fin()}. 

\lstinputlisting[style=program, caption={Barrier Synchronization
    ({\tt barrier})}, label=prg:barrier] {sync/examples/barrier.c}

Listing~\ref{prg:barrier} gives the third argument to the pip init()
function a non-NULL value this time. This is an additional method of
exporting a pointer to spawnd PiP jobs from the root. Children in this
example are given the address of the \pipstruct{pip_barrier_t} static
variable so they can synchronize by using \pipfunc{pip_barrier_wait()}.

To make the impact of the barrier synchronization more clear, the
synchronization only occurs when the second parameter of the program
execution is left blank; in this case, PiP tasks display the
\pipfunc{pip_gettime()} return values. The \pipfunc{pip_gettime()}
function returns the current value of the \linuxfunc{gettimeofday()}
function in double format with a second unit.

Listing~\ref{out:barrier} displays an example of this application
being run. The barrier synchronization is absent in the initial run,
and the values returned by \linuxfunc{gettimeofday()} exhibit
significant variation. A lesser difference can be detected in the
second run, where the barrier synchronization occurs.

\lstinputlisting[style=example,
  caption={Barrier Synchronization - Execution},
  label=out:barrier] {sync/examples/barrier.out}

\subsection{Using PThread Synchronization}

The PThread library's synchronization features for PiP tasks are
likewise available to users. This is due to the fact that PiP tasks
and threads both use the same address space. 

\subsection{\tt pthread_barrier}

The PThread's barrier functions can also be used to provide the same
barrier synchronization. Listing~\ref{prg:pthread-barrier} shows the
program's simple substitution of the PThread's barrier functions for the
PiP's barrier functions.

\lstinputlisting[style=program,
  caption={Pthread Barrier ({\tt
      pthread-barrier})},label=prg:pthread-barrier]
                {sync/examples/pthread-barrier.c}

\lstinputlisting[style=example, 
  caption={Pthread Barrier - Execution}, label=out:pthread-barrier]
                {sync/examples/pthread-barrier.out}
                
\subsection{\tt pthread_mutex}

Similarly, \linuxfunc{pthread_mutex} also works with PiP. 

\lstinputlisting[style=program,
  caption={Pthread Mutex ({\tt
      pthread-mutex})},label=prg:pthread-mutex]
                {sync/examples/pthread-mutex.c}

\lstinputlisting[style=example, 
  caption={Pthread Mutex - Execution}, label=out:pthread-mutex]
                {sync/examples/pthread-mutex.out}
